%%%%%
%%%%%  Naudokite LUALATEX, ne LATEX.
%%%%%
%%%%
\documentclass[]{VUMIFTemplateClass}

\usepackage{indentfirst}
\usepackage{amsmath, amsthm, amssymb, amsfonts}
\usepackage{mathtools}
\usepackage{physics}
\usepackage{graphicx}
\usepackage{verbatim}
\usepackage[hidelinks]{hyperref}
\usepackage{color,algorithm,algorithmic}
\usepackage[nottoc]{tocbibind}
\usepackage{tocloft}

\usepackage{titlesec}
\newcommand{\sectionbreak}{\clearpage}

\makeatletter
\renewcommand{\fnum@algorithm}{\thealgorithm}
\makeatother
\renewcommand\thealgorithm{\arabic{algorithm} algoritmas}

\usepackage{biblatex}
\bibliography{bibliografija}
%% norint pakeisti bibliografijos šaltinių numeravimą (skaitiniu arba raidiniu), pakeitimus atlikti VUMIFTemplateClass.cls 150 eilutėje

% Author's MACROS
\newcommand{\EE}{\mathbb{E}\,} % Mean
\newcommand{\ee}{{\mathrm e}}  % nice exponent
\newcommand{\RR}{\mathbb{R}}


\studijuprograma{Programų sistemų} %Studijų programą įrašyti kilmininko linksniu (pavyzdžiui – Programų sistemų, Finansų ir draudimų matematikos ir t. t.)
\darbotipas{Bakalauro baigiamojo darbo planas} % Bakalauro baigiamasis darbas arba magistro baigiamasis darbas
\darbopavadinimas{Maišymo proceso modeliavimas YAG reakcijose}
\darbopavadinimasantras{Modelling the mixing process in YAG reactions}
\autorius{Arnas Vaicekauskas}

%Autorių gali būti ir daugiau, tuo atveju, kiekvienas autorius rašomas iš naujos eilutės, ir pridedamas titulinis.tex arba dvigubasTitulinis.tex dokumentuose
%\antrasautorius{Vardas Pavardė} %Jei toks yra, kitu atveju ištrinti

\vadovas{Asist. Dr. Rokas Astrauskas}
\recenzentas{pedagoginis/mokslinis vardas Vardas Pavardė} %Jei toks yra žinomas, kitu atveju ištrinti

\begin{document}
\selectlanguage{lithuanian}

\onehalfspacing
\input{titulinis}

%% Padėkų skyrius
\section{Darbo planas}

\subsection{Tyrimo objektas ir aktualumas}

Itrio aliuminio granato (YAG) kristalai, legiruoti neodimio jonais, yra viena populiariausių medžiagų lazerių aktyviosioms terpėms, naudojamoms pramoninėje gamyboje ir medicinoje. Chemiškai paprasčiausias ir pramoninei gamybai labiausiai pritaikytas šios medžiagos sintezės metodas yra kietafazė reakcija, kurios spartinimui taikomas reagentų išmaišymas. Šiame darbe analizuojamas YAG kietafazės reakcijos modelis, į kurį įtrauktas išmaišymo mechanizmas.

\subsection{Darbo tikslas}

Šio darbo tikslas yra sudaryti kompiuterinę difuzijos-reakcijos sistemą, kuri modeliuotų kietafazę YAG reakciją bei ištirti maišymo modelio poveikį YAG kristalų sintezės spartai modeliuojant reakcija įvairaus dydžio erdvėse.

\subsection{Keliami uždaviniai ir laukiami rezultatai}
\subsubsection{Uždaviniai}
\begin{enumerate}
  
    \item Sudaryti kompiuterinius YAG reakcijos modelius remiantis baigtinių skirtumų skaitiniais metodais
    \item Patikrinti kompiuterinių modelių rezultatų korektiškumą bei palyginti modelių rezultatus tarpusavyje
    \item Apibrėžti medžiagų maišymo modelius
    \item Integruoti medžiagų maišymo modelius į skaitinius YAG reakcijos modelius
    \item Ištirti kompiuterinių modelių rezultatus įvairaus dydžio erdvėse
\end{enumerate}
\subsubsection{Laukiami rezultatai}
\begin{enumerate}
    \item Identifikuoti skaitiniai metodai, kurie bus naudojami YAG reakcijos modelio įgyvendinimui

    \item Įgyvendinti korektiškai veikiantys YAG reakcijos modeliai remiantis identifikuotais skaitiniais metodais

    \item Apibrėžti maišymo modeliai integruoti į kompiuterinius YAG reakcijos modelius

\end{enumerate}

\newpage

\subsection{Tyrimo metodai}

\begin{enumerate}
    \item \textbf{Modeliavimo metodas}. Kompiuteriniai modeliai tyrimui bus konstruojami šiais baigtinių skirtumų metodais:
    \begin{itemize}
        \item Išreikštinis FTCS (\textit{angl. explicit forward time, centered space}) metodas
        \item Neišreikštinis kintamosios krypties metodas (\textit{angl. alternating direction implicit, ADI}) 
    \end{itemize}
    \item \textbf{Auksinio pjūvio paieškos algoritmas}. Ieškomas optimalus medžiagų išmaišymo laikas naudojant auksinio pjūvio paieškos algoritmą.
    \item \textbf{Lyginamoji analizė}. Lyginami skirtingais skaitiniais metodai įgyvendintų kompiuterinių YAG reakcijos modelių rezultatai. 
\end{enumerate}


\subsection{Numatomas darbo atlikimo procesas}

\begin{enumerate}
    \item Analizuojama literatūra 
    \item Sudaromi kompiuteriai YAG reakcijos modeliai naudojant python programavimo kalbą su \textit{NumPy} bei \textit{SciPy} paketais
    \item Tikrinamas kompiuterinių YAG reakcijos modelių rezultatų korektiškumas, lyginami skirtingų kompiuterinių modelių rezultatai. Duomenų vizualizavimui naudojamas \texdtit{Matplotlib} paketas
    \item Apibrėžiami maišymo modeliai
    \item Maišymo poveikio bei korektiškumo tyrimas kompiuterinių YAG reakcijos modelių rezultatuose 
    \item Tiriamas maišymo proceso poveikis įvairaus dydžio erdvėse
\end{enumerate}

\subsection{Literatūros šaltinių apibūdinimas}

\cite{ivanauskasComputationalModellingYAG2009,ivanauskasModellingSolidState2005,mackeviciusCloserLookComputer2012} - pasiūlyta matematinė difuzijos-reakcijos sistema, kuria modeliuojama YAG sintezės reakcija bei išmatuotos fizinės konstantos prie skirtingų aplinkos sąlygų, kurios leidžia kiekybiškai lyginti kompiuterinių modelių rezultatus. Šis modelis bus naudojamas įgyvendinti kompiuterinį YAG reakcijos modelį. \cite{pressNumericalRecipes3rd2007,levequeFiniteDifferenceMethods2007} - aprašo skaitinius metodus, kuriais gali būti sprendžiamos parabolinių dalinių diferencialinių lygčių sistemos. Šie šaltiniais bus naudojami literatūros analizei bei jais bus vadovaujamasi įgyvendintant kompiuterinius modelius.

\printbibliography[title = {Literatūra ir šaltiniai}]

\end{document}
