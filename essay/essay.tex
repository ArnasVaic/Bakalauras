\documentclass[]{VUMIFTemplateClass}

\addbibresource{bibliography.bib}

\begin{document}

\thispagestyle{empty}

\begin{flushright}
Vilniaus Universitetas \\
Matematikos ir informatikos fakultetas \\
Priėmimo komisijai
\end{flushright}

\vspace{1cm}

Mano vardas Arnas Vaicekauskas, šiais metais įgijau informatikos mokslų bakalauro laipsnį studijuodamas programų sistemas. Studijose pavyko pasiekti puikius rezultatus ir įgyti \textit{Cum Laude} diplomą.

Studijuoti šią sritį ketinimai atsirado dar mokykloje, kai supratau, kad turiu pašaukimą tiksliesiems mokslams. Studijų metu 
supratau, kad mane ypač domina matematikos taikymas informatikoje. Antrame kurse atsirado galimybė pasirinkti neprivalomą ir ypač sudėtingą diferencialinių lygčių kursą, kuris nulėmė visas likusias studijas. Būtent čia įgijau bazines žinias, kurių reikia norint dirbti su matematiniais fizinius ar cheminius procesus apibūdinančiais modeliais. Įgavęs šias žinias nesunkiai pasirinkau baigiamojo darbo temą.

Vadovaujant Dr. Rokui Astrauskui baigiamajame darbe tyriau maišymo proceso modelius, pritaikytus itrio aliuminio granato (YAG) sintezės reakcijoje. Darbą apsigyniau aukščiausiam įvertinimui, o išvados pasirodė reikšmingos ne tik man, tačiau ir darbo vadovui bei recenzentui, todėl darbą pristačiau tarptautinėje konferencijoje \enquote{eStream 2025} ir kasmet vykstančioje Lietuvos matematikų draugijos konferencijoje \cite{LietuvosMatematikuDraugijos}.

Per minėtą diferencialinių lygčių kursą susipažinau su dėstytoju Raimundu Vidunu, su kuriuo dar kurso metu pradėjome dirbti prie teorinių skaičiavimų, kurie galiausiai prisidėjo prie mokslino straipsnio. Tyrėme neįprastą ląstelinio automato \enquote{Game of Life} versiją, kuri vyksta ant toro paviršiaus su maža tikimybe nepaklusti įprastoms taisyklėms \cite{vidunasConwaysGameLife2025}. Prie šio darbo prisidėjau atlikdamas įvairius skaičiavimus, kurie turėjo būti ypač optimizuoti norint apdoroti itin didelį duomenų kiekį. Darbas buvo pristatytas praeitų metų Lietuvos matematikų draugijos konferencijoje \cite{stikonasKONFERENCIJOSPROGRAMINISKOMITETAS2024} ir Simons Center for Geometry and Physics (SCGP) vykusioje konferencijoje, Niujorke \cite{MurmurationsArithmeticGeometry}.

Per studijas taip pat sukaupiau nemažai darbinės patirties -- nuo pirmo kurso pabaigos dirbu kaip sistemų inžinierius. Darbe tenka susidurti su įvairiausiomis technologijomis, tačiau daugiausia patirties įgijau su C\# programavimo kalba. Be technologinių iššūkių darbe taip pat teko išmokti kitų sistemų kūrimo aspektų -- poreikių analizės, reikalavimų inžinerijos, sistemų architektūros bei projektų valdymo.

Manau, kad studijuodamas kompiuterinį modeliavimą ugdysiu savo, kaip matematinio modeliavimo eksperto, kompetenciją ir įgysiu žinių, kurios man leis toliau tęsti darbus šioje srityje. Studijas noriu tęsti būtent šioje programoje, nes mano nuomone, jos geriau pritaikytos mano poreikiams negu programų sistemos.

\vspace{1em}

Pagarbiai \\
\vspace{0.5em}
\textbf{Arnas Vaicekauskas} 

\printbibliography

\end{document}
