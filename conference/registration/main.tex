\documentclass[a4paper,12pt]{article}
\usepackage[lithuanian]{babel}  % Lithuanian language support
\usepackage[T1]{fontenc}        % Proper font encoding
\usepackage{amsmath, amssymb}   % Math symbols and equations
\usepackage{graphicx}           % Images
\usepackage{hyperref}           % Clickable links
\usepackage{cite}               % Better citation formatting

\begin{document}

\title{\textbf{Rinkėjo modelis su periodinių apklausų mechanizmu}}
\author{%
    Rokas Astrauskas, Aleksejus Kononovičius, Marijus Radavičius, Feliksas Ivanauskas \\
    \textit{Vilniaus universitetas}
}
\date{} % No date
\maketitle

\begin{abstract}
Atsitiktinių procesų modeliai ir statistinė fizika gali būti panaudoti įvairių socialinių reiškinių tyrimams. Vienas populiariausių tokių modelių yra rinkėjo modelis, kuris šiame darbe pritaikome nagrinėdami apklausų įtaką rinkėjų nuomonei.
\end{abstract}

\section{Įvadas}
Sistemoje turime \(N\) agentų, kurie gali būti vienoje iš dviejų būsenų: \(0^\text{m}\) arba \(1^\text{m}\). 

\section{Matematinis modelis}
Apklausos atliekamos laiko \( \tau \) intervalais. Kiekvieno periodo pradžioje atliekama nauja apklausa, kurios rezultatas yra:
\[
A_k = X(k\tau),
\]
kur \( X(k\tau) \) žymi agentų skaičių būsenoje \(1^\text{m}\) apklausos atlikimo metu.

Agentai keičia būsenas pagal šiuos perėjimo spartas:
\[
\lambda_i^+ = (N - X) \left[\sigma_1 + h A_{k-1} \right], \quad
\lambda_i^- = X \left[\sigma_0 + h (N - A_{k-1}) \right].
\]

\section{Rezultatai}
Tyrime pateiktas naujas simuliavimo algoritmas, apjungiantis Gillespie ir Next reaction metodus.

\section{Išvados}
Modelio analizė rodo, kad rinkėjų nuomonė priklauso nuo apklausų rezultatų ir jų įtakos parametrų.

\begin{thebibliography}{9}
\bibitem{castellano2009}
Castellano, C., Fortunato, S., and Loreto, V. (2009). Statistical physics of social dynamics. \textit{Reviews of modern physics}, \textbf{81}(2), 591.

\bibitem{kononovicius2024}
Kononovičius, A., Astrauskas, R., Radavičius, M., and Ivanauskas, F. (2024). Delayed interactions in the noisy voter model through the periodic polling mechanism. \textit{arXiv preprint} \textbf{arXiv:2403.10277}.
\end{thebibliography}

\end{document}
