\documentclass{beamer}

\usepackage[version=4]{mhchem}  
\usepackage{tikz}
\usepackage{caption}
\usepackage{gensymb}
\usepackage{polyglossia}
\setmainlanguage{lithuanian}
\usepackage[
    backend=biber,
    style=numeric,
    sorting=ynt
]{biblatex}

\bibliography{bibliografija}

\usetheme{Copenhagen}

\setbeamertemplate{caption}[numbered]
\setbeamertemplate{headline}{}

\date{2025}

\title[]{Medžiagų maišymo modeliavimas YAG reakcijoje}
\subtitle{Modelling the mixing of reagents in YAG reaction}

\author[Arnas Vaicekauskas]
{
    A.~Vaicekauskas\inst{1}\\ 
    \small Darbo vadovas: Asist. Dr. R.~Astrauskas\inst{1}
}

\institute[MIF]
{
  \inst{1}
  Matematikos ir informatikos fakultetas\\
  Vilniaus Universitetas
}


\begin{document}

\frame{\titlepage}

\frame{
    \tableofcontents[currentsubsection,subsectionstyle=show]
}

\section{YAG sintezės reakcija}

\begin{frame}
    \frametitle{YAG sintezės reakcija}

    \centering
    \begin{align*}
        \ce{3Y_2O_3 + 5Al_2O_3 -> 2Y_3Al_5O_{12}}
    \end{align*}    

    \begin{itemize}
        \item Reakcija vykdoma kaitinant itrio ir aliuminio oksidų homogeninį mišinį
        \item Sintezė vyksta $1000-1600 C\degree$ laipsnių temperatūroje
        \item Reakcija gali užtrukti keliolika valandų
        \item Praktikoje reagentai yra periodiškai išmaišomi norint paspartinti reakciją 
    \end{itemize}

\end{frame}

\section{Motyvacija}

\begin{frame}
\frametitle{Motyvacija}
 
    \begin{itemize}
        \item Medžiagų maišymo modeliavimas gali padėti suprasti procesus vykstančius YAG reakcijos metu, kurių nėra galimybės ištirti fiziškai dėl aplinkos sąlygų reikalingų reakcijai
        \item Kompiuterinių modelių efektyvumas leidžia simuliuoti YAG sintezės reakciją daug greičiau ir pigiau negu fiziniai eksperimentai laboratorijoje
    \end{itemize}

\end{frame}

\section{Tyrimo tikslas}
\begin{frame}
    \frametitle{Tyrimo tikslas}
    \begin{itemize}
        \item Ištirti kompiuterinio maišymo modelio poveikį reakcijos rezultatams modeliuojant skirtingo dydžio sritis
        \item Pasirinkti tokį kompiuterinį maišymo modelį, kurio poveikis rezultatams atitiktų fizinius eksperimentus
    \end{itemize}
\end{frame}

\section{Matematinis modelis}


\subsection{Reakcijos-difuzijos sistema}

\begin{frame}
\frametitle{Matematinė reakcijos-difuzijos sistema}

YAG Reakcijos--difuzijos sistema

\begin{align*}
    \frac{\partial c_1}{\partial t} & =-3kc_1c_2+D\left(\frac{\partial^2c_1}{\partial x^2}+\frac{\partial^2c_1}{\partial y^2}\right) \\
    \frac{\partial c_2}{\partial t} & =-5kc_1c_2+D\left(\frac{\partial^2c_2}{\partial x^2}+\frac{\partial^2c_2}{\partial y^2}\right)\\
    \frac{\partial c_3}{\partial t} & =2kc_1c_2\\
    (x, y, t)&\in(0, W)\times(0, H)\times[0, T]
\end{align*}

\end{frame}

\subsection{Pradinės ir kraštinės sąlygos}

\begin{frame}
    \frametitle{Pradinės ir kraštinės sąlygos}
    \begin{figure}
        \centering
        \begin{tikzpicture}[scale=2.0]
            \draw[fill=white] (0,0) rectangle (1,1);
            \draw[fill=white] (1,1) rectangle (2,2);
            \draw[fill=gray!50] (0,1) rectangle (1,2);
            \draw[fill=gray!50] (1,0) rectangle (2,1);
    
            % Draw the boundary of the square
            \draw[thick] (0,0) rectangle (2,2);
    
            % Draw axes
            \draw[->] (-0.5, 0) -- (2.5, 0) node[anchor=north west] {$x$};
            \draw[->] (0, -0.5) -- (0, 2.5) node[anchor=north east] {$y$};
    
            % Mark the origin
            \node[anchor=north east] at (0,0) {$(0, 0)$};
    
            % Mark the side length
            \draw[-] (2,0) -- (2,2);
            \draw[-] (0,2) -- (2,2);

            \node at (2.1, 0.75) {$H$};
            \node at (0.75, 2.1) {$W$};
            
            \draw (0.5, 1.5) node[anchor=center] {$5c_0$};
            \draw (1.5, 0.5) node[anchor=center] {$5c_0$};
            \draw (1.5, 1.5) node[anchor=center] {$3c_0$};
            \draw (0.5, 0.5) node[anchor=center] {$3c_0$};

            % boundary conditions
            \draw[->] (-0.05,1) -- (-0.55,1);
            \node at (-0.55, 1.5) {$\frac{dc_m}{dx} = 0$};

            \draw[->] (2.05,1) -- (2.55,1);
            \node at (2.55, 1.5) {$\frac{dc_m}{dx} = 0$};

            \draw[->] (1, -0.05) -- (1, -0.55);
            \node at (1.5, -0.25) {$\frac{dc_m}{dy} = 0$};

            \draw[->] (1, 2.05) -- (1, 2.55);
            \node at (1.5, 2.25) {$\frac{dc_m}{dy} = 0$};

        \end{tikzpicture}
        \caption{Pradinės ir kraštinės sąlygos modeliui.}
        \label{initial-boundary-condition}
    \end{figure}
\end{frame}

\subsection{Stabilumo sąlyga}

\begin{frame}
    \frametitle{Stabilumo sąlyga}
    Galima parodyti, kad skaitinis sprendinys bus stabilus tada, kai tenkinama ši nelygybė:
    \begin{align*}
        \Delta t \leqslant \left(15kc_0+2D\left((\Delta x)^{-2}+(\Delta y)^{-2}\right)\right)^{-1}
    \end{align*}
    Čia $k$ - reakcijos greitis, $D$ - difuzijos konstanta, $\Delta x, \Delta y$ - diskretūs erdvės žingsniai, $c_0$ - pradinių medžiagų koncentracijų didžiausias bendras daliklis.
\end{frame}

\subsection{Modelio rezultatai}

\begin{frame}
    \frametitle{Reakcijos--difuzijos modelio rezultatai}
    Modelio evoliucija laike su pradinėmis ir kraštinėmis sąlygomis \eqref{initial-boundary-condition}
    \centering
    \includegraphics[width=12cm]{assets/example-0.png} \\ 
    \includegraphics[width=12cm]{assets/example-2.png}

\end{frame}

\section{Maišymo modeliai}
\subsection{Reakcijos stabdymo sąlyga}

\begin{frame}
    \frametitle{Reakcijos stabdymo sąlyga}
    \begin{align*}
        q(t_\text{stop})=0.03q(0)
    \end{align*}
    Čia $q(t)$ yra pirmos ir antros medžiagų kiekis sistemoje. Reakcija stabdoma tada, kai pirmų dviejų medžiagų kiekis sistemoje pasiekia 3\% pradinio kiekio.
\end{frame}

\subsection{Atsitiktinis maišymas}

\begin{frame}
\frametitle{Atsitiktinis maišymas}
\begin{figure}
\centering
\begin{tikzpicture}[scale=1.5]

    \fill[gray!30] (0,1) rectangle (1, 2);
    \fill[gray!30] (1,0) rectangle (2, 1);

    % Original Grid
    \draw[thick] (0,0) rectangle (2,2);
    \draw[thick] (1,0) -- (1,2);
    \draw[thick] (0,1) -- (2,1);

    \node at (0.5,1.5) {$\Omega_1$};
    \node at (1.5,1.5) {$\Omega_2$};
    \node at (0.5,0.5) {$\Omega_3$};
    \node at (1.5,0.5) {$\Omega_4$};

    % Arrow
    \draw[->, thick] (2.5,1) -- (3.5,1);

    % Transformed Grid
    \begin{scope}[shift={(4,0)}]

        \fill[gray!30] (0,0) rectangle (1, 1);
        \fill[gray!30] (1,1) rectangle (2, 2);

        \draw[thick] (0,0) rectangle (2,2);
        \draw[thick] (1,0) -- (1,2);
        \draw[thick] (0,1) -- (2,1);

        \node at (0.5,1.5) {\rotatebox{270}{$\Omega_3$}}; % Rotated 180° horizontally
        \node at (1.5,1.5) {$\Omega_1$};             % No change
        \node at (0.5,0.5) {\rotatebox{180}{$\Omega_4$}}; % Upside down
        \node at (1.5,0.5) {\rotatebox{90}{$\Omega_2$}};  % 90° rotation
    \end{scope}
\end{tikzpicture}
\caption{Atsitiktinio išmaišymo metu metu reakcijos erdvės sritys yra atsitiktinai pasukamos ir susikeičiamos vietomis. }
\label{random-mix}
\end{figure}
\end{frame}

\subsection{Tobulas maišymas}
\begin{frame}
\frametitle{Tobulas maišymas}
\begin{figure}
\centering
\begin{tikzpicture}[scale=1.5]
    % Original Grid
    \fill[gray!30] (0,1) rectangle (1, 2);
    \fill[gray!30] (1,0) rectangle (2, 1);
    \draw[<->] (0.75,0.75) -- (1.25,1.25);
    \draw[<->] (1.25,0.75) -- (0.75,1.25);
    \draw[thick] (0,0) rectangle (2,2);
    \draw[dashed] (1,0) -- (1,2);
    \draw[dashed] (0,1) -- (2,1);

    \node at (0.5,1.5) {$\Omega_1$};
    \node at (1.5,1.5) {$\Omega_2$};
    \node at (0.5,0.5) {$\Omega_3$};
    \node at (1.5,0.5) {$\Omega_4$};

    % Arrow
    \draw[->, thick] (2.5,1) -- (3.5,1);

    % Transformed Grid
    \begin{scope}[shift={(4,0)}]
        \fill[gray!30] (0,1) rectangle (1, 2);
        \fill[gray!30] (1,0) rectangle (2, 1);
        
        \draw[dashed] (0,0) rectangle (2,2);
        \draw[thick] (1,0) -- (1,2);
        \draw[thick] (0,1) -- (2,1);

        \node at (0.5,1.5) {$\Omega_4$};
        \node at (1.5,1.5) {$\Omega_3$};
        \node at (0.5,0.5) {$\Omega_2$};
        \node at (1.5,0.5) {$\Omega_1$};
    \end{scope}
\end{tikzpicture}
\caption{Tobulo išmaišymo metu metu reakcijos erdvės sritys yra susikeičiamos vietomis įstrižai, tokiu būdu reakcijos greitis padidėja daugiausiai. }
\label{perfect-mix}
\end{figure}
\end{frame}

\section{Modelių rezultatai}

\subsection{Atsitiktinio maišymo rezultatai}
\begin{frame}
    \frametitle{Atsitiktinio maišymo rezultatai}
    \begin{figure}
        \centering
        \includegraphics[width=12cm]{assets/random-mix-example-c0-1.png} \\
        \includegraphics[width=12cm]{assets/random-mix-example-c2-1.png}
        \caption{Modelio evoliucija laike su pradinėmis ir kraštinėmis sąlygomis \eqref{initial-boundary-condition}, kai naudojamas atsitiktinio išmaišymo modelis \eqref{random-mix}.}
    \end{figure}
\end{frame}

\begin{frame}
    \frametitle{Atsitiktinio maišymo rezultatai}
    \begin{figure}
        \centering
        \includegraphics[width=7cm]{assets/bad-mix-qnt-compare-1.png}
        \caption{Atsitiktinio išmaišymo modelio \eqref{random-mix} problema - gali ženkliai prailginti reakcijos pabaigos laiką. Tokio modelio poveiki reikia matuoti statistiniu bandymu arba pilnu perrinkimu. }
    \end{figure}
\end{frame}

\subsection{Tobulo maišymo rezultatai}
\begin{frame}
\frametitle{Tobulo maišymo rezultatai}
\begin{figure}
\centering
\includegraphics[width=7cm]{assets/optimal-mix-qnt-1.png}
\caption{Tobulas maišymas deterministinis ir tikroviškiau atspindi išmaišymo pasekmes.}
\end{figure}
\end{frame}

\begin{frame}
    \frametitle{Tobulo maišymo rezultatai}
    \begin{figure}
        \centering
        \includegraphics[width=7cm]{assets/mix-end-1.png}
        \caption{Reakcijos pabaigos laiko priklausomybė nuo išmaišymo laiko, kai naudojamas tobulo išmaišymo modelis \eqref{perfect-mix}.}
    \end{figure}
\end{frame}

\subsection{Tobulas maišymo modelis didesnėms erdvėms}
\begin{frame}
    \frametitle{Maišymas didesnėje erdvėje}
    \begin{figure}
    \centering
    \begin{tikzpicture}
        % Original Grid
    
        \fill[gray!30] (1, 0) rectangle (3, 1);
        \fill[gray!30] (1, 3) rectangle (3, 4);
    
        \fill[gray!30] (0, 1) rectangle (1, 3);
        \fill[gray!30] (3, 1) rectangle (4, 3);
    
        \draw[thick] (0,0) rectangle (2,2);
        \draw[thick] (0,2) rectangle (2,4);
        \draw[thick] (2,0) rectangle (4,2);
        \draw[thick] (2,2) rectangle (4,4);
        \draw[dashed] (1,0) -- (1,4);
        \draw[dashed] (0,1) -- (4,1);
        \draw[dashed] (3,0) -- (3,4);
        \draw[dashed] (0,3) -- (4,3);
    
        \draw[<->] (0.75,0.75) -- (1.25,1.25);
        \draw[<->] (1.25,0.75) -- (0.75,1.25);
    
        \draw[<->] (2.75,0.75) -- (3.25,1.25);
        \draw[<->] (3.25,0.75) -- (2.75,1.25);
    
        \draw[<->] (0.75,2.75) -- (1.25,3.25);
        \draw[<->] (1.25,2.75) -- (0.75,3.25);
    
        \draw[<->] (2.75,2.75) -- (3.25,3.25);
        \draw[<->] (3.25,2.75) -- (2.75,3.25);
    
        \node at (0.5,3.5) {$\Omega_1$};
        \node at (1.5,3.5) {$\Omega_2$};
        \node at (0.5,2.5) {$\Omega_5$};
        \node at (1.5,2.5) {$\Omega_6$};
    
        \node at (2.5,3.5) {$\Omega_3$};
        \node at (3.5,3.5) {$\Omega_4$};
        \node at (2.5,2.5) {$\Omega_7$};
        \node at (3.5,2.5) {$\Omega_8$};
    
        \node at (0.5,1.5) {$\Omega_9$};
        \node at (1.5,1.5) {$\Omega_{10}$};
        \node at (0.5,0.5) {$\Omega_{13}$};
        \node at (1.5,0.5) {$\Omega_{14}$};
    
        \node at (2.5,1.5) {$\Omega_{11}$};
        \node at (3.5,1.5) {$\Omega_{12}$};
        \node at (2.5,0.5) {$\Omega_{15}$};
        \node at (3.5,0.5) {$\Omega_{16}$};
    
        % Arrow
        \draw[->, thick] (4.5,2) -- (5.5,2);
    
        % Transformed Grid
        \begin{scope}[shift={(6,0)}]
            \fill[gray!30] (1, 0) rectangle (3, 1);
            \fill[gray!30] (1, 3) rectangle (3, 4);
    
            \fill[gray!30] (0, 1) rectangle (1, 3);
            \fill[gray!30] (3, 1) rectangle (4, 3);
    
            \draw[dashed] (0,0) rectangle (4,4);
    
            \draw[dashed] (2,0) -- (2,4);
            \draw[dashed] (0,2) -- (4,2);
    
            \draw[thick] (1,0) -- (1,4);
            \draw[thick] (0,1) -- (4,1);
            \draw[thick] (3,0) -- (3,4);
            \draw[thick] (0,3) -- (4,3);
    
            \node at (0.5,3.5) {$\Omega_6$};
            \node at (1.5,3.5) {$\Omega_5$};
            \node at (0.5,2.5) {$\Omega_2$};
            \node at (1.5,2.5) {$\Omega_1$};
    
            \node at (2.5,3.5) {$\Omega_8$};
            \node at (3.5,3.5) {$\Omega_7$};
            \node at (2.5,2.5) {$\Omega_4$};
            \node at (3.5,2.5) {$\Omega_3$};
    
            \node at (0.5,1.5) {$\Omega_{14}$};
            \node at (1.5,1.5) {$\Omega_{13}$};
            \node at (0.5,0.5) {$\Omega_{10}$};
            \node at (1.5,0.5) {$\Omega_{9}$};
    
            \node at (2.5,1.5) {$\Omega_{16}$};
            \node at (3.5,1.5) {$\Omega_{15}$};
            \node at (2.5,0.5) {$\Omega_{12}$};
            \node at (3.5,0.5) {$\Omega_{11}$};
        \end{scope}
    \end{tikzpicture}
    \caption{Tobulo maišymo modelis didesnėje erdvėje analogiškas modeliui \eqref{perfect-mix} atkartojant jį periodiškai.}
    \label{large-perfect-mix}
\end{figure}
\end{frame}

\subsection{Tobulas maišymo modelio rezultatai didesnėje erdvėje}
\begin{frame}
    \frametitle{Maišymas didesnėje erdvėje}
    \begin{figure}
    \centering
    \includegraphics[width=7cm]{assets/mix-end-large-1.png}
    \caption{Reakcijos pabaigos laiko priklausomybė nuo išmaišymo laiko, kai naudojamas tobulo išmaišymo modelis didesnėje erdvėje \eqref{large-perfect-mix}.}
    \end{figure}
\end{frame}

% \begin{frame}
% \frametitle{Rezultatai}
% \begin{itemize}
%     \item Sukurtas kompiuterinis YAG reakcijos modelis. 
%     \item Teoriškai parodyta skaitinio modelio stabilumo sąlyga
%     \item Kompiuterinio modelio rezultatai buvo analizuojami ir buvo užtikrinta, kad modelis veikia korektiškai
%     \item Pasiūlyti du maišymo modeliai - atsitiktinis ir tobulas
%     \item Išmaišymo modeliai integruoti į kompiuterinį YAG reakcijos modelį
%     \item Atlikta papildyto kompiuterinio modelio rezultatų analizė
% \end{itemize}
% \end{frame}

\begin{frame}
\frametitle{Išvados}
\begin{itemize}
    \item Atsitiktinio maišymo modelio rezultatai neatitinka realybėje pastebimų rezultatų, kai reakcija modeliuojama mažoje srityje, kurioje susiduria tik 4-ios mikrodalelės. Norint iš šio modelio išgauti tikrovę atitinkančius rezultatus yra būtina modeliuoti didesnę erdvės sritį.

    \item Tobulo išmaišymo modelio rezultatai atitinka realybėje pastebimą reakcijos pagreitėjimą.
    
    \item Modeliuojant didesnę erdvės sritį, tobulo išmaišymo modelio rezultatai kinta gana nežymiai, todėl maišymo modeliavimui užtenka modeliuoti mažą reakcijos erdvės sritį su 4-iom skirtingų medžiagų mikrodalelėmis

\end{itemize}
\end{frame}

% \begin{frame}
% \frametitle{Literatūros šaltiniai}
% \printbibliography
% \end{frame}

\end{document}