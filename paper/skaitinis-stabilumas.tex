\subsection{Modelio skaitinis stabilumas}

Norint užtikrinti skaitinį programos stabilumą, reikia užtikrinti, kad visais laiko momentais, visuose diskretizuotos erdvės taškuose, visų medžiagų koncentracijos išliktų ne neigiamos. Šiai sąlygai išpildyti, užtenka pasirinkti pakankamai mažą laiko žingsnį $\Delta t$. Pirmiausia įvedame porą konstantų:
\begin{align*}
\mu_x = \frac{D\Delta t}{(\Delta x)^2}, \quad
\mu_y = \frac{D\Delta t}{(\Delta y)^2}
\end{align*}
Tada pertvarkome dviejų dimensijų skaitinį modelį \eqref{numerical-eqs} taip, kad kairėse lygčių pusėse liktų medžiagų koncentracija laiko momentu $n+1$, o dešinėse lygčių pusėse sugrupuojame narius pagal medžiagų koncentraciją skirtinguose diskretizuotos erdvės taškuose:
\begin{subequations} \label{eqs:r-coefs}
  \begin{align}
  c^{n+1}_{1,i,j}&=\underbrace{(1-3k\Delta tc^{n}_{2,i,j}-2(\mu_x+\mu_y))}_{R_1}c^n_{1,i,j}+\mu_xc^n_{1,i-1,j}+\mu_xc^n_{1,i+1,j}+\mu_yc^n_{1,i,j-1}+\mu_yc^n_{1,i,j+1} \label{r-coefs1}\\
  c^{n+1}_{2,i,j}&=\underbrace{(1-5k\Delta tc^{n}_{1,i,j}-2(\mu_x+\mu_y))}_{R_2}c^n_{2,i,j}+\mu_xc^n_{2,i-1,j}+\mu_xc^n_{2,i+1,j}+\mu_yc^n_{2,i,j-1}+\mu_yc^n_{2,i,j+1} \label{r-coefs2}\\
  c^{n+1}_{3,i,j}&=c^n_{3,i,j}+2k\Delta tc^{n}_{1,i,j}c^{n}_{2,i,j} \label{r-coefs3}
  \end{align}
\end{subequations}
Baziniu atveju, kai $n=0$, medžiagų koncentracija visuose taškuose yra ne neigiama, kaip numatyta pradinėse sąlygose \eqref{intial-cond}. Darome indukcijos hipotezės prielaidą, kad medžiagų koncentracija visuose diskretizuotos erdvės taškuose, laiko momentu $n$ bus ne neigiama:
\begin{align} \label{induction-assumption}
  c^n_{m,i,j} \geqslant 0, \quad m=1,2,3,\quad i=0,1,\dots,N-1,\quad j=0,1,\dots,M-1
\end{align}
Akivaizdu, kad lygtyje \eqref{r-coefs3}, medžiagos koncentracija $c^{n+1}_{3,i,j}$ bus ne neigiama:
\begin{align*}
  \Delta t > 0 \land c^n_{m,i,j}\geqslant 0 \implies c^{n+1}_{3,i,j}=c^n_{3,i,j}+2\Delta tc^{n}_{1,i,j}c^{n}_{2,i,j}\geqslant 0 
\end{align*}
Pirmos medžiagos lygtyje \eqref{r-coefs1}, galima pastebėti, kad dėmenys su medžiagų koncentracijomis iš aplinkinių diskretizuotos erdvės taškų visada bus ne neigiami dėl prielaidos \eqref{induction-assumption} ir fakto, kad $\mu_x>0$ ir $\mu_y>0$:
\begin{align*}
  \mu_xc^n_{1,i-1,j}+\mu_xc^n_{1,i+1,j}+\mu_yc^n_{1,i,j-1}+\mu_yc^n_{1,i,j+1}\geqslant 0
\end{align*}
\newpage
Taigi, $c^{n+1}_{1,i,j}$ ženklą lemia tik koeficientas $R_1$, todėl įvedame ribojimą, kad $R_1\geqslant 0$. Analogiškai, iš antros medžiagos lygties \eqref{r-coefs2} gauname, kad $R_2\geqslant 0$ ir turime nelygybių sistemą:
\begin{align} \label{impl-ineqs}
  \begin{cases}
    1-3k\Delta tc^{n}_{2,i,j}-2(\mu_x+\mu_y)\geqslant 0\\
    1-5k\Delta tc^{n}_{1,i,j}-2(\mu_x+\mu_y)\geqslant 0
  \end{cases}, \quad i=0,1,\dots,N-1, \quad j=0,1,\dots,M-1
\end{align}
Išreiškę nelygybes \eqref{impl-ineqs} per laiko žingsnį $\Delta t$ gauname:
\begin{align} \label{dt-ineq}
  \begin{cases}
    \Delta t \leqslant (3kc^{n}_{2,i,j}+2D((\Delta x)^{-2}+(\Delta y)^{-2}))^{-1}\\
    \Delta t \leqslant (5kc^{n}_{1,i,j}+2D((\Delta x)^{-2}+(\Delta y)^{-2}))^{-1}
  \end{cases}
\end{align}
Gautas nelygybes galima apjungti dėl jų panašios struktūros. Norint, kad apjungta nelygybė tenkintų sistemą \eqref{dt-ineq}, reikia išrinkti mažiausią įmanomą laiko žingsnį $\Delta t$, o taip bus tada, kai trupmenos vardiklis bus kuo įmanoma didesnis. Dėl to gauta nelygybė įgaus formą:
\begin{align}
  \Delta t \leqslant \left(\max(3kc^{n}_{2,i,j}, 5kc^{n}_{1,i,j})+2D\left((\Delta x)^{-2}+(\Delta y)^{-2}\right)\right)^{-1}
\end{align}
Taigi, parodėme, kad su pakankamai mažu laiko žingsniu $\Delta t$ išvengiame neigiamų sprendinio reikšmių. Tačiau čia sustoti būtų nenaudinga, nes turime rekursyvią priklausomybę -- norint pasirinkti $\Delta t$, reikia žinoti maksimalią medžiagų reikšmę simuliacijoje, o jai sužinoti reikia atlikti simuliaciją su pasirinktu laiko žingsniu $\Delta t$.

Parodysime, kad galima panaikinti laiko žingsnio priklausomybę nuo laiko momento $n$ ir kad $\Delta t$ priklauso tik nuo pradinių sąlygų $c^0_{1,i,j}$ ir $c^0_{2,i,j}$. Medžiagos kiekį sistemoje galima gauti integravus medžiagos koncentraciją erdvėje:

\begin{align}
  q&=\int_\Omega c dV \label{quantity-general}
\end{align}

Iš čia galime išvesti išraišką medžiagos kiekio pokyčiui per laiką:

\begin{align}
  \frac{\partial q}{\partial t}=\frac{\partial}{\partial t}\int_\Omega c dV=\int_\Omega\frac{\partial c}{\partial t}dV,
\end{align}

Įstatome pirmų dviejų medžiagų lygtis iš matematinio modelio \eqref{rect} ir gauname lygtis pirmos $q_1$ ir antros $q_2$ medžiagos kiekio pokyčiams per laiką:
\begin{align}
  \frac{\partial q_1}{\partial t}=-3k\int_\Omega c_1c_2\,dV + D\int_\Omega \Delta c_1\,dV=-3k\int_\Omega c_1c_2\,dV + D\int_\Omega \nabla \cdot (\nabla c_1)\,dV\\
  \frac{\partial q_2}{\partial t}=-5k\int_\Omega c_1c_2\,dV + D\int_\Omega \Delta c_2\,dV=-5k\int_\Omega c_1c_2\,dV + D\int_\Omega \nabla \cdot (\nabla c_2)\,dV
\end{align}
\newpage
Pagal Gauso-Ostrogradskio divergencijos teoremą ir kraštinę sąlygą \eqref{general-boundary-cond} gauname, kad:
\begin{align}\label{no-q-change}
\int_\Omega \nabla \cdot (\nabla c_m) dV = \int_{\partial\Omega} \nabla c_k \cdot \vec{n}\, dS = 0,\quad m=1,2
\end{align}
Todėl pirmos ir antros medžiagų kiekio pokytis per laiką bus ne teigiamas:
\begin{subequations} \label{negative-quantity}
\begin{align}
  \frac{\partial q_1}{\partial t}=-3k\int_\Omega c_1c_2\,dV \leqslant 0\\
  \frac{\partial q_2}{\partial t}=-5k\int_\Omega c_1c_2\,dV\leqslant 0
\end{align}
\end{subequations}
Tai reiškia, kad maksimalios pirmos ir antros medžiagų reikšmės bus laiko momente $n=0$, tad nelygybę galime suprastinti iki šios formos:
\begin{align}
  \Delta t \leqslant \left(\max(3kc^{0}_{2,i,j}, 5kc^{0}_{1,i,j})+2D\left((\Delta x)^{-2}+(\Delta y)^{-2}\right)\right)^{-1}
\end{align}
Laiko žingsnis čia vis dar priklauso nuo diskretizuotos erdvės koordinatės $(i, j)$, tačiau nesunku pastebėti, kad užtenka parinkti didžiausią reikšmę iš kiekvienos pradinės sąlygos -- tokiu būdu laiko žingsnis gausis mažiausias. Iš pradinės sąlygos \eqref{intial-cond} turime:
\begin{align*}
\max c^0_{1,i,j}=3c_0,\quad i=0,1,\dots,N-1, \quad j=0,1,\dots,M-1\\
\max c^0_{2,i,j}=5c_0,\quad i=0,1,\dots,N-1, \quad j=0,1,\dots,M-1
\end{align*}
Taigi, kad simuliacija išliktų skaitiškai stabili, reikia, kad laiko žingsnis $\Delta t$ tenkintų šią nelygybę.
\begin{align}
  \Delta t \leqslant \left(15kc_0+2D\left((\Delta x)^{-2}+(\Delta y)^{-2}\right)\right)^{-1}\label{numerical-stability-condition}
\end{align}

