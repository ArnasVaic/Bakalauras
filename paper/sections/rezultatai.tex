\sectionnonum{Rezultatai ir išvados}

\subsection*{Rezultatai}

Šiame darbe buvo įgyvendinti šie uždaviniai:

\begin{itemize}
    \item Sudaryti kompiuteriniai YAG reakcijos modeliai remiantis išreikštiniu ir ADI skaitiniais metodais
    \item Teoriškai parodyta išreikštinio skaitinio modelio stabilumo sąlyga
    \item Sukurta kintamo laiko žingsnio strategija, kuri leidžia efektyviai modeliuoti YAG sintezės reakciją didelėse srityse naudojant ADI metodą
    \item Modelių rezultatai buvo verifikuojami ir buvo užtikrinta, kad modeliai veikia korektiškai 
    \item Išmatuotas bei palygintas abiejų kompiuterinių modelių tikslumas ir efektyvumas
    \item Atsitiktinio ir tobulo maišymo modeliai integruoti į kompiuterinius (YAG) reakcijos modelius
    \item Ištirtas maišymo modelių poveikis įvairaus dydžio erdvėse ir nustatyta kaip skirtingų maišymo modelių savybės priklauso nuo erdvės dydžio
    \item Nustatyta reakcijos trukmės priklausomybė nuo maišymo momento, jos kaita didinant modeliuojamą erdvę. Paieškos algoritmu nustatyti optimalūs maišymo momentai.
\end{itemize}

\subsection*{Išvados}

Iš rezultatų analizės galima daryti šias išvadas:

\begin{itemize}
    
    \item Modeliuojant tobulą išmaišymą YAG reakcijoje, optimalus maišymo laikas išlieka resikšmingai nepakitęs, nepriklausomai nuo modeliuojamos srities dydžio todėl norint tiksliai apskaičiuoti optimalų maišymo laiką užtenka modeliuoti erdvės sritį, kuri yra vienos mikrodalelės dydžio

    \item Modeliuojant tobulą išmaišymą YAG reakcijoje, didinant modeliuojamą sritį, optimali reakcijos trukmė artėja prie ribinės reikšmės ir palyginus su bendra reakcijos trukme skirtumas labai mažas, tai reiškia, kad užtenka modeliuoti maža erdvės sritį ($1\mu m^2$)

    \item Tobulo maišymo modelis atspindi savybes, kuriomis pasižymi realus maišymo procesas. Jei maišymas vyksta reakcijos pradžioje arba pabaigoje nepastebime jokio pokyčio reakcijos trukmėje, tačiau maišant kitais laiko momentais pastebime, kad reakcijos trukmė trumpėja. Šis maišymo modelis yra tinkamas modeliuoti išmaišymo procesą

    \item Kintamo laiko žingsnio strategija leidžia ypač efektyviai spręsti YAG reakcijos uždavinį, o įtaka sprendinio tikslumui yra nereikšminga

    \item Atsitiktinis maišymo modeliavimas neturi savybių, kuriomis turėtų pasižymėti bet koks maišymo modelis -- beveik visuomet reakcijos trukmė pailgėja. Tai galioja tada, kai modeliuojama erdvė yra dviejų dimensijų ir yra nedidelė palyginus su tikraja reakcijos erdve. Rezultatai rodo, kad net 64 kartus padidintoje srityje (originali sritis yra vienos mikrodalelės dydžio) vidutinis reakcijos laikas pailgėja daugiau nei 40 valandų, todėl laikome, kad atsitiktinis maišymo modelis yra nepraktiškas

\end{itemize}