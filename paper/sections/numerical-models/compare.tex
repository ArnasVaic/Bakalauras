\subsection{Skaitinių modelių analizė}

% Kas bus šioje sekcijoje?
% - Modelių tikslumo lyginimas
% - Modelių efektyvumo lyginimas

\subsubsection*{Skaitinių modelių efektyvumas}

Viena iš priežasčių, kodėl buvo pasirinktas ADI metodas, yra jo efektyvumas. Nors ir šio metodo pritaikymas nagrinėjamai sistemai nėra idealus -- reakcijos komponentai yra laikomi išreikštinais ir todėl besąlygiškas stabilumas nėra garantuojamas, naudojant šį metodą turėtume gebėti didinti laiko žingsnį $\Delta t$ ir taip sumažinti iteracijų skaičių. Pirmiausią palyginsime įgyvendintų skaitinių modelių efektyvumą, kai naudojamas laiko žingsnis $\Delta t$ yra toks pat.

\begin{figure}[h!]
  \centering
  \includegraphics[width=0.75\textwidth]{images/adi/adi-iteration-time.png}
  \caption{FTCS ir ADI metodų sprendimo laiko priklausomybė nuo diskrečių taškų skaičiaus reakcijos erdvėje. Kiekvienai skirtingai rezoliucijai buvo parinkta toks laiko žingsnis, kuris užtikrintų stabilumą.}
  \label{fig:adi-iteration-time}
\end{figure}