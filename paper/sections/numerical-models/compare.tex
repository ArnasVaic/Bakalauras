\subsection{Skaitinių modelių analizė}

% Kas bus šioje sekcijoje?
% - Modelių tikslumo lyginimas
% - Modelių efektyvumo lyginimas

\subsubsection*{Praktinis skaitinių modelių efektyvumas}

Norint ištirti skirtingų metodų efektyvumą, paleisime skirtingus skaitinius modelius su tais pačiais parametrais ir didinsime reakcijos erdvės rezoliuciją -- padidėjęs diskrečių taškų skaičius padidins ir skaičiavimų trukmę, todėl galėsim palyginti metodų efektyvumą iš praktinės pusės. Laiko žingsniai skaitiniams modeliams parinkti skirtingai -- FTCS metodui buvo pasirinktas toks žingsnis, kuris užtikrintų stabilumą, o ADI metodui buvo taikoma anksčiau apibūdinta kintamo laiko žingsnio strategija \eqref{alg:scgq}.

\begin{figure}[h!]
  \centering
  \includegraphics[width=0.5\textwidth]{images/ftcs-adi-perf.png}
  \caption{FTCS ir ADI metodų sprendimo laiko priklausomybė nuo diskrečių taškų skaičiaus reakcijos erdvėje.}
  \label{fig:adi-ftcs-solve-time-comparison}
\end{figure}

\ref{fig:adi-ftcs-solve-time-comparison}-ame pavyzdyje matome, kad pakankamai žemos rezoliucijos konfigūracijoms išreikštinis FTCS skaitinis modelis sprendinį apskaičiuoja greičiau negu ADI, tačiau konfigūracijoms, kurių rezoliucija viršija $100\times100$, sprendinį greičiau randa ADI skaitinis modelis. Vadovaudamiesi šiais rezultatais tolimesniems skaičiavimams ir maišymo modeliavimui didesnėse erdvėse galime naudoti ADI modelį.