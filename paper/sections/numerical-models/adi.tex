\newpage
\subsection{Neišreikštinis kintamos krypties metodas}

Neišreikštinį kintamos krypties metodą (\textit{alternating direction implicit, ADI}) JAV mokslininkai Donald W. Peaceman ir Henry H. Rachford Jr. pristatė savo straipsnyje \enquote{Skaitinis sprendinys parabolinėms ir elipsinės diferencialinėms lygtims} \cite{doi:10.1137/0103003}. Nuo to laiko šis metodas plačiai taikomas matematinio modeliavimo srityje. Kaip galima suprasti iš straipsnio pavadinimo, šis metodas pritaikytas spręsti parabolines ir elipsines diferencialinių lygčių sistemas. Kadangi mūsų nagrinėjamą sistemą sudaro parabolinės diferencialinės lygtys, jį ir taikysime.

Šis metodas yra tarpinis variantas tarp išreikštinio FTCS ir Crank-Nicholson metodo, kuriuo bandoma išlaikyti sprendimo greitį ir tikslumą. Įprastoms parabolinėms lygtims šis metodas būna besąlygiškai stabilus \cite{liAlternatingDirectionImplicit2021}, tai leidžia pasirinkti bet kokio dydžio laiko žingsnį ir tokiu būdu sumažinti bendrą laiko žingsnių kiekį, kurį reikia įvykdyti. Toliau pritaikysime metodą nagrinėjamai sistemai.

Užuot tiesiogiai skaičiavę skaitinį sprendinį sekančiu laiko momentu $c^{n+1}_{m,i,j}$, pirmiausia apskaičiuosime tarpinį sprendinį, kurį žymėsime $c^*_{m,i,j}$. Skaitiniam modeliui sudaryti naudosime tas pačias išvestinių aproksimacijas, kurias naudojome FTCS metode, tačiau pagal ADI metodo specifiką laikysime, kad difuzijos komponenta sudaro išreikštinė ir neišreikštinė dalys -- \hbox{t. y.} $x$ ašies išvestinei skaičiuoti bus naudojamas ateinančio laiko žingsnio sprendinys $c^*$, o $y$ ašies išvestinei naudosime jau turimą sprendinį $c^n$. Dėl priežasčių, kurios bus atskleistos vėliau, reakcijos komponentus visada skaičiuosime pagal jau turimą laiko žingsnį $c^n$. Sudarome lygtis tarpiniam sprendiniui $c^*$ pasinaudojant standartiniu baigtinių skirtumų operatorių žymėjimu, kurį naudojome ir FTCS modeliui $\delta_x^2[c_{ij}]=c_{i-1,j}-2c_{i,j}+c_{i+1,j}$, $\delta_y^2[c_{ij}]=c_{i,j-1}-2c_{i,j}+c_{i,j+1}$:
\begin{subequations} \label{eqs:adi-half-step}
\begin{align}
	\frac{c^*_{1,i,j} - c^n_{1,i,j}}{\frac{1}{2}\Delta t} &= D_1 \left( \frac{\delta_x^2[c^*_{1,i,j}]}{(\Delta x)^2} + \frac{\delta_y^2[c^n_{1,i,j}]}{(\Delta y)^2} \right) - 3kc^n_{1,i,j}c^n_{2,i,j}\\
	\frac{c^*_{2,i,j} - c^n_{2,i,j}}{\frac{1}{2}\Delta t} &= D_2 \left( \frac{\delta_x^2[c^*_{2,i,j}]}{(\Delta x)^2} + \frac{\delta_y^2[c^n_{2,i,j}]}{(\Delta y)^2} \right) - 5kc^n_{1,i,j}c^n_{2,i,j}\\
	\frac{c^*_{3,i,j} - c^n_{3,i,j}}{\frac{1}{2}\Delta t} &= D_3 \left( \frac{\delta_x^2[c^*_{3,i,j}]}{(\Delta x)^2} + \frac{\delta_y^2[c^n_{3,i,j}]}{(\Delta y)^2} \right) +2kc^n_{1,i,j}c^n_{2,i,j}
\end{align}
\end{subequations}

Analogiškas lygtis yra sudaromos sprendiniui $c^{n+1}$ rasti, tačiau apkeičiame sprendinius, kuriuos naudojame difuzijos komponentui -- $x$ ašies išvestinei naudosime jau turimą sprendinį $c^{*}$, o $y$ ašies naudosime sekančio laiko žingsnio sprendinį $c^{n+1}$:

\begin{subequations} \label{eqs:adi-next-step}
\begin{align}
	\frac{c^{n+1}_{1,i,j} - c^*_{1,i,j}}{\frac{1}{2}\Delta t} 
	&= D_1 \left( \frac{\delta_x^2[c^{*}_{1,i,j}]}{(\Delta x)^2} 
	+ \frac{\delta_y^2[c^{n+1}_{1,i,j}]}{(\Delta y)^2} \right) - 3kc^*_{1,i,j}c^*_{2,i,j}\\
	\frac{c^{n+1}_{2,i,j} - c^*_{2,i,j}}{\frac{1}{2}\Delta t} 
	&= D_2 \left( \frac{\delta_x^2[c^*_{2,i,j}]}{(\Delta x)^2}
	+ \frac{\delta_y^2[c^{n+1}_{2,i,j}]}{(\Delta y)^2} \right) - 5kc^*_{1,i,j}c^*_{2,i,j}\\
	\frac{c^{n+1}_{3,i,j} - c^*_{3,i,j}}{\frac{1}{2}\Delta t} 
	&= D_3 \left( \frac{\delta_x^2[c^*_{3,i,j}]}{(\Delta x)^2} 
	+ \frac{\delta_y^2[c^{n+1}_{3,i,j}]}{(\Delta y)^2} \right) +2kc^*_{1,i,j}c^*_{2,i,j}
\end{align}
\end{subequations}

\newpage

Kitaip nei FTCS metodo atveju, sprendinys, kurį bandome rasti, priklauso ne tik nuo praėjusio laiko žingsnio sprendinio, bet ir nuo paties savęs. Tai galioja tiek tarpiniam, tiek sekančio žingsnio sprendiniui. Rezultate mes gauname lygčių sistemą, kurią reikia spręsti. Iš pradžių persitvarkysime gautas lygtis tarpiniam sprendiniui $c^*$, susikeldami skirtingų laiko žingsnių sprendinių komponentus į skirtingas lygybės puses. Kadangi lygtis yra labai panašios viena į kitą jas pertvarkysime bendrai žymėdami medžiagos indeksą $m$, reakcijos koeficientus $\alpha_1 = -3$, $\alpha_2 = -5$, $\alpha_3 = 2$. Taip pat naudosime žymėjimą konstantoms: $\mu_{mx} = \frac{\Delta t D_m}{2(\Delta x)^2}$, $\mu_{my} = \frac{\Delta t D_m}{2(\Delta y)^2}$, $\mu_m = \frac{1}{2} \Delta t \alpha_m k$.

\begin{align*}
  \frac{c^{*}_{m,i,j} - c^n_{m,i,j}}{\frac{1}{2}\Delta t} 
  &= D_m \left( \frac{\delta_x^2[c^{*}_{m,i,j}]}{(\Delta x)^2} 
  + \frac{\delta_y^2[c^n_{m,i,j}]}{(\Delta y)^2} \right) 
  + \alpha_mkc^*_{1,i,j}c^*_{2,i,j} \\
  c^*_{m,i,j} 
  &= \frac{1}{2}\Delta t D_m \left( \frac{\delta_x^2[c^{*}_{m,i,j}]}{(\Delta x)^2} 
  + \frac{\delta_y^2[c^n_{m,i,j}]}{(\Delta y)^2} \right)
  + \frac{1}{2}\Delta t \alpha_m kc^n_{1,i,j}c^n_{2,i,j} + c^n_{m,i,j}\\
  c^*_{m,i,j} - \underbrace{\frac{\Delta t D_m}{2(\Delta x)^2}}_{\mu_{mx}}\delta_x^2[c^{*}_{m,i,j}] 
  &= \underbrace{\frac{\Delta t D_m}{2(\Delta y)^2}}_{\mu_{my}}\delta_y^2[c^n_{m,i,j}]
  + \underbrace{\frac{1}{2}\Delta t \alpha_m k}_{\mu_m}c^n_{1,i,j}c^n_{2,i,j} + c^n_{m,i,j}\\
  c^*_{m,i,j} - \mu_{mx}\delta_x^2[c^{*}_{m,i,j}]
  &= \mu_{my}\delta_y^2[c^n_{m,i,j}] + \mu_m c^n_{1,i,j}c^n_{2,i,j} + c^n_{m,i,j}
\end{align*}

Išskleidę paskutinė lygtyje esančius baigtinių skirtumų operatorius gauname lygtį tarpiniam sprendiniui $c^*$:

\begin{align}
  -\mu_{mx}c^{*}_{m,i-1,j}+(1+2\mu_{mx})c^{*}_{m,i,j}-\mu_{mx}c^{*}_{m,i+1,j}
  &= \mu_{my}c^n_{m,i,j-1}+(1-2\mu_{my})c^n_{m,i,j}+\mu_{my}c^n_{m,i,j+1}+\mu_m c^n_{1,i,j}c^n_{2,i,j}
\end{align}

Analogiškai galima išvesti lygtis sekančiam laiko žingsniui $c^{n+1}_{m,i,j}$:

\begin{align}
  % c^{n+1}_{m,i,j} - \mu_{my}\delta_y^2[c^{n+1}_{m,i,j}]
  % &= \mu_{mx}\delta_x^2[c^*_{m,i,j}] + \mu_m c^*_{1,i,j}c^*_{2,i,j} + c^*_{m,i,j}\\
  -\mu_{my}c^{n+1}_{m,i,j-1}+(1+2\mu_{my})c^{n+1}_{m,i,j}-\mu_{my}c^{n+1}_{m,i,j+1}
  &= \mu_{mx}c^*_{m,i-1,j}+(1-2\mu_{mx})c^*_{m,i,j}+\mu_{mx}c^*_{m,i+1,j}+\mu_m c^*_{1,i,j}c^*_{2,i,j}
\end{align}

Kraštinėms sąlygoms apskaičiuoti naudosime centrinę pirmosios išvestinės aproksimaciją:

\begin{align*}
  \frac{\partial c}{\partial x}\Big|_{x=x_i, y=y_j, t=t_n} 
  &\approx \frac{c^n_{i+1,j}-c^n_{i-1,j}}{2\Delta x} &
  \frac{\partial c}{\partial y}\Big|_{x=x_i, y=y_j, t=t_n} 
  &\approx \frac{c^n_{i,j+1}-c^n_{i,j-1}}{2\Delta y}
\end{align*}

Įstate šias aproksimacijas į modelio kraštines sąlygas \eqref{boundary-cond} gauname:

\begin{subequations} \label{boundary-cond-approx}
\begin{align} 
  c_{i+1,j} &= c_{i-1,j}, \text{ kai } i = 0 \text{ arba } i = N-1 &
  c_{i,j+1} &= c_{i,j-1}, \text{ kai } j = 0 \text{ arba } j = M-1 &
\end{align}
\end{subequations}

Turint kraštines sąlygas \eqref{boundary-cond-approx} bei lygtis \eqref{eqs:adi-half-step} ir \eqref{eqs:adi-next-step} galime užrašyti sistemas matricos pavidalu. Pirmiausia užrašome lygčių sistemas tarpiniam sprendiniui $c^*$:

\begin{align} \label{eqs:adi-system-half}
  \resizebox{\textwidth}{!}{$
  \begin{bmatrix}
    1 + 2\mu_{mx} & -2\mu_{mx} & 0 & \cdots & 0 & 0\\
    -\mu_{mx} & 1 + 2\mu_{mx} & -\mu_{mx} & \cdots & 0 & 0\\
    0 & -\mu_{mx} & 1 + 2\mu_{mx} & \cdots & 0 & 0\\
    \vdots & \vdots & \vdots & \ddots & \vdots & \vdots\\
    0 & 0 & 0 & \cdots & 1 + 2\mu_{mx} & -\mu_{mx}\\
    0 & 0 & 0 & \cdots & -2\mu_{mx} & 1 + 2\mu_{mx}
  \end{bmatrix}
  \begin{bmatrix}
    c^*_{m,0,j}\\
    c^*_{m,1,j}\\
    c^*_{m,2,j}\\
    \vdots\\
    c^*_{m,N-1,j}
  \end{bmatrix}
  =
  \begin{bmatrix}
    \mu_{my}c^n_{m,0,j-1}+(1-2\mu_{my})c^n_{m,0,j}+\mu_{my}c^n_{m,0,j+1}+\mu_m c^n_{1,0,j}c^n_{2,0,j}\\
    \mu_{my}c^n_{m,1,j-1}+(1-2\mu_{my})c^n_{m,1,j}+\mu_{my}c^n_{m,1,j+1}+\mu_m c^n_{1,1,j}c^n_{2,1,j}\\
    \mu_{my}c^n_{m,2,j-1}+(1-2\mu_{my})c^n_{m,2,j}+\mu_{my}c^n_{m,2,j+1}+\mu_m c^n_{1,2,j}c^n_{2,2,j}\\
    \vdots\\
    \mu_{my}c^n_{m,N-1,j-1}+(1-2\mu_{my})c^n_{m,N-1,j}+\mu_{my}c^n_{m,N-1,j+1}+\mu_m c^n_{1,N-1,j}c^n_{2,N-1,j}
  \end{bmatrix}
  $}
\end{align}

Lygčių sistemos sekančio laiko žingsnio sprendiniui $c^{n+1}$:

\begin{align} \label{eqs:adi-system-next}
  \resizebox{\textwidth}{!}{$
  \begin{bmatrix}
    1 + 2\mu_{my} & -2\mu_{my} & 0 & \cdots & 0 & 0\\
    -\mu_{my} & 1 + 2\mu_{my} & -\mu_{my} & \cdots & 0 & 0\\
    0 & -\mu_{my} & 1 + 2\mu_{my} & \cdots & 0 & 0\\
    \vdots & \vdots & \vdots & \ddots & \vdots & \vdots\\
    0 & 0 & 0 & \cdots & 1 + 2\mu_{my} & -\mu_{my}\\
    0 & 0 & 0 & \cdots & -2\mu_{my} & 1 + 2\mu_{my}
  \end{bmatrix}
  \begin{bmatrix}
    c^*_{m,i,0}\\
    c^*_{m,i,1}\\
    c^*_{m,i,2}\\
    \vdots\\
    c^*_{m,i,M-1}
  \end{bmatrix}
  =
  \begin{bmatrix}
    \mu_{mx}c^n_{m,i-1,0}+(1-2\mu_{mx})c^n_{m,i,0}+\mu_{mx}c^n_{m,i+1,0}+\mu_m c^n_{1,i,0}c^n_{2,i,0}\\
    \mu_{mx}c^n_{m,i-1,1}+(1-2\mu_{mx})c^n_{m,i,1}+\mu_{mx}c^n_{m,i+1,1}+\mu_m c^n_{1,i,1}c^n_{2,i,1}\\
    \mu_{mx}c^n_{m,i-1,2}+(1-2\mu_{mx})c^n_{m,i,2}+\mu_{mx}c^n_{m,i+1,2}+\mu_m c^n_{1,i,2}c^n_{2,i,2}\\
    \vdots\\
    \mu_{mx}c^n_{m,i-1,M-1}+(1-2\mu_{mx})c^n_{m,i,M-1}+\mu_{mx}c^n_{m,i+1,M-1}+\mu_m c^n_{1,i,M-1}c^n_{2,i,M-1}
  \end{bmatrix}
  $}
\end{align}

Šios sistemos sprendinys yra atitinkamai ieškomo laiko žingsnio sprendinio eilutė arba stulpelis, todėl kiekvieno laiko žingsnio sprendiniui rasti šią sistemą reikės spręsti $M + N$ kartų. Taip pat galima pastebėti, kad kairėje lygties pusėje esančios matricos yra tridiagonalinės, todėl jas galima spręsti efektyviai naudojant tridiagonalinės matricos algoritmą. Būtent dėl šios priežasties mes pasirinkome laikyti reakcijos komponentus išreikštinais, kitu atveju matricos nebūtų tridiagonalinės. Šis sprendimas turi ir savo minusų -- kadangi reakcijos dėmuo laikomas išreikštiniu, sistema nebėra besąlygiškai stabili.
