\newpage
\subsection{Neišreikštinis kintamos krypties metodas}

Neišreikštinį kintamos krypties metodą JAV mokslininkai Donaldas W.~Peacmanas ir Henry~H.~Rachfordas jaunesnysis pristatė savo straipsnyje \enquote{Skaitinis sprendinys parabolinėms ir elipsinėms diferencialinėms lygtims}~\cite{doi:10.1137/0103003}. Nuo to laiko šis metodas plačiai taikomas matematinio modeliavimo srityje. Kaip galima spręsti iš straipsnio pavadinimo, metodas skirtas spręsti parabolinių ir elipsinių diferencialinių lygčių sistemas. Kadangi mūsų nagrinėjamąją sistemą sudaro parabolinės diferencialinės lygtys, šį metodą ir taikysime.

Šis metodas yra tarpinis variantas tarp išreikštinio ir Cranko-Nicolsono metodo, siekiant suderinti sprendinio greitį ir tikslumą. Įprastoms parabolinėms lygtims šis metodas yra besąlygiškai stabilus~\cite{liAlternatingDirectionImplicit2021}, todėl galima pasirinkti bet kokio dydžio laiko žingsnį ir taip sumažinti bendrą vykdomų žingsnių skaičių. Toliau šį metodą pritaikysime nagrinėjamai sistemai.

Užuot tiesiogiai skaičiavę skaitinį sprendinį sekančiu laiko momentu $c^{n+1}_{m,i,j}$, pirmiausia apskaičiuosime tarpinį sprendinį, kurį žymėsime $c^*_{m,i,j}$. Skaitiniam modeliui sudaryti naudosime tas pačias išvestinių aproksimacijas, kurias naudojome išreikštiniame metode, tačiau pagal ADI metodo specifiką laikysime, kad difuzijos komponentą sudaro išreikštinė ir neišreikštinė dalys -- \hbox{t. y.} $x$ ašies išvestinei skaičiuoti bus naudojamas ateinančio laiko žingsnio sprendinys $c^*$, o $y$ ašies išvestinei naudosime jau turimą sprendinį $c^n$. Dėl priežasčių, kurios bus atskleistos vėliau, reakcijos komponentus visada skaičiuosime pagal jau turimą laiko žingsnį $c^n$. Sudarome lygtis tarpiniam sprendiniui $c^*$ pasinaudojant standartiniu baigtinių skirtumų operatorių žymėjimu, kurį naudojome ir išreikštiniam modeliui $\delta_x^2[c_{ij}]=c_{i-1,j}-2c_{i,j}+c_{i+1,j}$, $\delta_y^2[c_{ij}]=c_{i,j-1}-2c_{i,j}+c_{i,j+1}$:
\begin{subequations} \label{eqs:adi-half-step}
\begin{align}
	\frac{c^*_{1,i,j} - c^n_{1,i,j}}{\frac{1}{2}\Delta t} &= D_1 \left( \frac{\delta_x^2[c^*_{1,i,j}]}{(\Delta x)^2} + \frac{\delta_y^2[c^n_{1,i,j}]}{(\Delta y)^2} \right) - 3kc^n_{1,i,j}c^n_{2,i,j}\\
	\frac{c^*_{2,i,j} - c^n_{2,i,j}}{\frac{1}{2}\Delta t} &= D_2 \left( \frac{\delta_x^2[c^*_{2,i,j}]}{(\Delta x)^2} + \frac{\delta_y^2[c^n_{2,i,j}]}{(\Delta y)^2} \right) - 5kc^n_{1,i,j}c^n_{2,i,j}\\
	\frac{c^*_{3,i,j} - c^n_{3,i,j}}{\frac{1}{2}\Delta t} &= D_3 \left( \frac{\delta_x^2[c^*_{3,i,j}]}{(\Delta x)^2} + \frac{\delta_y^2[c^n_{3,i,j}]}{(\Delta y)^2} \right) +2kc^n_{1,i,j}c^n_{2,i,j}
\end{align}
\end{subequations}

Analogiškos lygtys sudaromos sprendiniui $c^{n+1}$ rasti, tačiau sukeičiamas sprendinių, naudojamų difuzijos komponentui, taikymas: $x$ ašies išvestinei naudojamas jau apskaičiuotas tarpinis sprendinys $c^{*}$, o $y$ ašies išvestinei — sekančio laiko žingsnio sprendinys $c^{n+1}$:

\begin{subequations} \label{eqs:adi-next-step}
\begin{align}
	\frac{c^{n+1}_{1,i,j} - c^*_{1,i,j}}{\frac{1}{2}\Delta t} 
	&= D_1 \left( \frac{\delta_x^2[c^{*}_{1,i,j}]}{(\Delta x)^2} 
	+ \frac{\delta_y^2[c^{n+1}_{1,i,j}]}{(\Delta y)^2} \right) - 3kc^*_{1,i,j}c^*_{2,i,j}\\
	\frac{c^{n+1}_{2,i,j} - c^*_{2,i,j}}{\frac{1}{2}\Delta t} 
	&= D_2 \left( \frac{\delta_x^2[c^*_{2,i,j}]}{(\Delta x)^2}
	+ \frac{\delta_y^2[c^{n+1}_{2,i,j}]}{(\Delta y)^2} \right) - 5kc^*_{1,i,j}c^*_{2,i,j}\\
	\frac{c^{n+1}_{3,i,j} - c^*_{3,i,j}}{\frac{1}{2}\Delta t} 
	&= D_3 \left( \frac{\delta_x^2[c^*_{3,i,j}]}{(\Delta x)^2} 
	+ \frac{\delta_y^2[c^{n+1}_{3,i,j}]}{(\Delta y)^2} \right) +2kc^*_{1,i,j}c^*_{2,i,j}
\end{align}
\end{subequations}

\newpage

Kitaip nei išreikštinio metodo atveju, sprendinys, kurį bandome rasti, priklauso ne tik nuo praėjusio laiko žingsnio sprendinio, bet ir nuo paties savęs. Tai galioja tiek tarpiniam, tiek sekančio laiko žingsnio sprendiniui. Dėl to susidaro lygčių sistema, kurią reikia išspręsti. Pirmiausia persitvarkysime gautas lygtis tarpiniam sprendiniui $c^*$, atskirdami skirtingų laiko žingsnių sprendinių komponentus į skirtingas lygybės puses. Kadangi lygtys yra labai panašios tarpusavyje, jas pertvarkysime bendrai, pažymėdami medžiagos indeksą $m$ ir reakcijos koeficientus: $\alpha_1 = -3$, $\alpha_2 = -5$, $\alpha_3 = 2$. Taip pat įsivesime žymėjimus konstantoms: $\mu_{mx} = \frac{\Delta t D_m}{2(\Delta x)^2}$, $\mu_{my} = \frac{\Delta t D_m}{2(\Delta y)^2}$, $\mu_m = \frac{1}{2} \Delta t \alpha_m k$.

\begin{align*}
  \frac{c^{*}_{m,i,j} - c^n_{m,i,j}}{\frac{1}{2}\Delta t} 
  &= D_m \left( \frac{\delta_x^2[c^{*}_{m,i,j}]}{(\Delta x)^2} 
  + \frac{\delta_y^2[c^n_{m,i,j}]}{(\Delta y)^2} \right) 
  + \alpha_mkc^*_{1,i,j}c^*_{2,i,j} \\
  c^*_{m,i,j} 
  &= \frac{1}{2}\Delta t D_m \left( \frac{\delta_x^2[c^{*}_{m,i,j}]}{(\Delta x)^2} 
  + \frac{\delta_y^2[c^n_{m,i,j}]}{(\Delta y)^2} \right)
  + \frac{1}{2}\Delta t \alpha_m kc^n_{1,i,j}c^n_{2,i,j} + c^n_{m,i,j}\\
  c^*_{m,i,j} - \underbrace{\frac{\Delta t D_m}{2(\Delta x)^2}}_{\mu_{mx}}\delta_x^2[c^{*}_{m,i,j}] 
  &= \underbrace{\frac{\Delta t D_m}{2(\Delta y)^2}}_{\mu_{my}}\delta_y^2[c^n_{m,i,j}]
  + \underbrace{\frac{1}{2}\Delta t \alpha_m k}_{\mu_m}c^n_{1,i,j}c^n_{2,i,j} + c^n_{m,i,j}\\
  c^*_{m,i,j} - \mu_{mx}\delta_x^2[c^{*}_{m,i,j}]
  &= \mu_{my}\delta_y^2[c^n_{m,i,j}] + \mu_m c^n_{1,i,j}c^n_{2,i,j} + c^n_{m,i,j}
\end{align*}

Išskleidę paskutinėje lygties dalyje esančius baigtinių skirtumų operatorius, gauname lygtį tarpiniam sprendiniui $c^*$:

\begin{align} \label{eqs:adi-short-half}
  -\mu_{mx}c^{*}_{m,i-1,j}+(1+2\mu_{mx})c^{*}_{m,i,j}-\mu_{mx}c^{*}_{m,i+1,j}
  &= \mu_{my}c^n_{m,i,j-1}+(1-2\mu_{my})c^n_{m,i,j}+\mu_{my}c^n_{m,i,j+1}+\mu_m c^n_{1,i,j}c^n_{2,i,j}
\end{align}

Analogiškai galima išvesti lygtis sekančiam laiko žingsniui $c^{n+1}_{m,i,j}$:

\begin{align} \label{eqs:adi-short-next}
  -\mu_{my}c^{n+1}_{m,i,j-1}+(1+2\mu_{my})c^{n+1}_{m,i,j}-\mu_{my}c^{n+1}_{m,i,j+1}
  &= \mu_{mx}c^*_{m,i-1,j}+(1-2\mu_{mx})c^*_{m,i,j}+\mu_{mx}c^*_{m,i+1,j}+\mu_m c^*_{1,i,j}c^*_{2,i,j}
\end{align}

Kraštinėms sąlygoms apskaičiuoti naudosime centrinę pirmosios išvestinės aproksimaciją:

\begin{align*}
  \frac{\partial c}{\partial x}\Big|_{x=x_i, y=y_j, t=t_n} 
  &\approx \frac{c^n_{i+1,j}-c^n_{i-1,j}}{2\Delta x},\quad
  \text{ kai } i = 0 \text{ arba } i = N - 1
  \\
  \frac{\partial c}{\partial y}\Big|_{x=x_i, y=y_j, t=t_n} 
  &\approx \frac{c^n_{i,j+1}-c^n_{i,j-1}}{2\Delta y},\quad
  \text{ kai } j = 0 \text{ arba } j = M - 1
\end{align*}

Įstate šias aproksimacijas į modelio kraštines sąlygas \eqref{boundary-cond} gauname:

\begin{subequations} \label{boundary-cond-approx}
\begin{align} 
  c_{i+1,j} &= c_{i-1,j}, \text{ kai } i = 0 \text{ arba } i = N-1 \\
  c_{i,j+1} &= c_{i,j-1}, \text{ kai } j = 0 \text{ arba } j = M-1
\end{align}
\end{subequations}

\newpage

Vizualiai galime įsivaizduoti, kad diskrečiame tinklelyje, kurį apibrėžėme anksčiau, egzistuoja taškai \enquote{vaiduokliai} \cite{cocoFinitedifferenceGhostpointMultigrid2013}, kuriuose medžiagų koncentracija įgyja tokią pat reikšmę kaip ir taškuose, esančiuose per vieną diskretų erdvės žingsnį nuo srities kraštinės, kaip pavaizduota \ref{fig:ghost-points}-ame pavyzdyje.

\begin{figure}[h!]
  \centering
  \begin{tikzpicture}[every node/.style={font=\small},>=Stealth]

  % Draw points and labels below
  \node[draw,circle,fill=gray!60,inner sep=2pt] (a) at (0,0) {};
  \node[below=2pt of a] {$c_{m,-1,j}$};

  \node[draw,circle,inner sep=2pt] (b) at (2,0) {};
  \node[below=2pt of b] {$c_{m,0,j}$};

  \node[draw,circle,inner sep=2pt] (c) at (4,0) {};
  \node[below=2pt of c] {$c_{m,1,j}$};

  \node at (6,0) {$\cdots$};

  \node[draw,circle,inner sep=2pt] (d) at (8,0) {};
  \node[below=2pt of d] {$c_{m,N-2,j}$};

  \node[draw,circle,inner sep=2pt] (e) at (10,0) {};
  \node[below=2pt of e] {$c_{m,N-1,j}$};

  \node[draw,circle,fill=gray!60,inner sep=2pt] (f) at (12,0) {};
  \node[below=2pt of f] {$c_{m,N,j}$};

  % Draw arrows
  \draw[->,thick] (a) to[out=60,in=120] (c);
  \draw[->,thick] (f) to[out=120,in=60] (d);

  \end{tikzpicture}
  \caption{ADI metodo baigtinių skirtumų šablonas (\textit{angl. stencil}) modeliuojamos srities $x$ ašies kraštinėms. Analogiškai gali būtų pavaizduoti tokį šabloną $y$ ašies kraštinėms. }
  \label{fig:ghost-points}
\end{figure}
Turint kraštines sąlygas \eqref{boundary-cond-approx} bei lygtis \eqref{eqs:adi-short-half} ir \eqref{eqs:adi-short-next} galime pritaikyti tas pačias lygtis srities kraštinėms, pavyzdžiui, tarpinio sprendinio formulės pirmajai eilutei atrodytų štai taip:

\begin{align*}
  & (1 + 2\mu_{mx})c^{*}_{m,0,0} - 2\mu_{mx}c^{*}_{m,1,0} \\
  & \quad = (1 - 2\mu_{my})c^n_{m,0,0} + 2\mu_{my}c^n_{m,0,1} \\
  & \qquad + \mu_m c^n_{1,0,0}c^n_{2,0,0}, \quad \text{kai } i = 0 \\[1.5ex]
  %
  & -\mu_{mx}c^{*}_{m,i-1,0} + (1 + 2\mu_{mx})c^{*}_{m,i,0} - \mu_{mx}c^{*}_{m,i+1,0} \\
  & \quad = (1 - 2\mu_{my})c^n_{m,i,0} + 2\mu_{my}c^n_{m,i,1} \\
  & \qquad + \mu_m c^n_{1,i,0}c^n_{2,i,0}, \quad \text{kai } i = 1, 2, \dots, N-2 \\[1.5ex]
  %
  & (1 + 2\mu_{mx})c^{*}_{m,N-1,0} - 2\mu_{mx}c^{*}_{m,N-2,0} \\
  & \quad = (1 - 2\mu_{my})c^n_{m,N-1,0} + 2\mu_{my}c^n_{m,N-1,1} \\
  & \qquad + \mu_m c^n_{1,N-1,0}c^n_{2,N-1,0}, \quad \text{kai } i = N-1
\end{align*}


Šias lygtis galime apibendrinti ir užrašyti matricos pavidalu. Pirmiausia užrašome lygčių sistemas tarpiniam sprendiniui $c^*$:
\begin{align} \label{eqs:adi-system-half}
  \resizebox{\textwidth}{!}{$
  \begin{bmatrix}
    1 + 2\mu_{mx} & -2\mu_{mx} & 0 & \cdots & 0 & 0\\
    -\mu_{mx} & 1 + 2\mu_{mx} & -\mu_{mx} & \cdots & 0 & 0\\
    0 & -\mu_{mx} & 1 + 2\mu_{mx} & \cdots & 0 & 0\\
    \vdots & \vdots & \vdots & \ddots & \vdots & \vdots\\
    0 & 0 & 0 & \cdots & 1 + 2\mu_{mx} & -\mu_{mx}\\
    0 & 0 & 0 & \cdots & -2\mu_{mx} & 1 + 2\mu_{mx}
  \end{bmatrix}
  \begin{bmatrix}
    c^*_{m,0,j}\\
    c^*_{m,1,j}\\
    c^*_{m,2,j}\\
    \vdots\\
    c^*_{m,N-1,j}
  \end{bmatrix}
  =
  \begin{bmatrix}
    \mu_{my}c^n_{m,0,j-1}+(1-2\mu_{my})c^n_{m,0,j}+\mu_{my}c^n_{m,0,j+1}+\mu_m c^n_{1,0,j}c^n_{2,0,j}\\
    \mu_{my}c^n_{m,1,j-1}+(1-2\mu_{my})c^n_{m,1,j}+\mu_{my}c^n_{m,1,j+1}+\mu_m c^n_{1,1,j}c^n_{2,1,j}\\
    \mu_{my}c^n_{m,2,j-1}+(1-2\mu_{my})c^n_{m,2,j}+\mu_{my}c^n_{m,2,j+1}+\mu_m c^n_{1,2,j}c^n_{2,2,j}\\
    \vdots\\
    \mu_{my}c^n_{m,N-1,j-1}+(1-2\mu_{my})c^n_{m,N-1,j}+\mu_{my}c^n_{m,N-1,j+1}+\mu_m c^n_{1,N-1,j}c^n_{2,N-1,j}
  \end{bmatrix}
  $}
\end{align}

Lygčių sistemos sekančio laiko žingsnio sprendiniui $c^{n+1}$:

\begin{align} \label{eqs:adi-system-next}
  \resizebox{\textwidth}{!}{$
  \begin{bmatrix}
    1 + 2\mu_{my} & -2\mu_{my} & 0 & \cdots & 0 & 0\\
    -\mu_{my} & 1 + 2\mu_{my} & -\mu_{my} & \cdots & 0 & 0\\
    0 & -\mu_{my} & 1 + 2\mu_{my} & \cdots & 0 & 0\\
    \vdots & \vdots & \vdots & \ddots & \vdots & \vdots\\
    0 & 0 & 0 & \cdots & 1 + 2\mu_{my} & -\mu_{my}\\
    0 & 0 & 0 & \cdots & -2\mu_{my} & 1 + 2\mu_{my}
  \end{bmatrix}
  \begin{bmatrix}
    c^*_{m,i,0}\\
    c^*_{m,i,1}\\
    c^*_{m,i,2}\\
    \vdots\\
    c^*_{m,i,M-1}
  \end{bmatrix}
  =
  \begin{bmatrix}
    \mu_{mx}c^n_{m,i-1,0}+(1-2\mu_{mx})c^n_{m,i,0}+\mu_{mx}c^n_{m,i+1,0}+\mu_m c^n_{1,i,0}c^n_{2,i,0}\\
    \mu_{mx}c^n_{m,i-1,1}+(1-2\mu_{mx})c^n_{m,i,1}+\mu_{mx}c^n_{m,i+1,1}+\mu_m c^n_{1,i,1}c^n_{2,i,1}\\
    \mu_{mx}c^n_{m,i-1,2}+(1-2\mu_{mx})c^n_{m,i,2}+\mu_{mx}c^n_{m,i+1,2}+\mu_m c^n_{1,i,2}c^n_{2,i,2}\\
    \vdots\\
    \mu_{mx}c^n_{m,i-1,M-1}+(1-2\mu_{mx})c^n_{m,i,M-1}+\mu_{mx}c^n_{m,i+1,M-1}+\mu_m c^n_{1,i,M-1}c^n_{2,i,M-1}
  \end{bmatrix}
  $}
\end{align}

Šios sistemos sprendinys yra atitinkamo laiko žingsnio sprendinio eilutė arba stulpelis, todėl kiekvienam laiko žingsniui rasti šią sistemą reikės spręsti $M + N$ kartų. Taip pat galima pastebėti, kad kairėje lygties pusėje esančios matricos yra tridiagonalinės, todėl jas galima spręsti efektyviai, naudojant tridiagonalinės matricos algoritmą. Būtent dėl šios priežasties pasirinkome laikyti reakcijos komponentus išreikštinais -- kitu atveju matricos nebūtų tridiagonalinės. Šis sprendimas turi ir trūkumų -- kadangi reakcijos dėmuo laikomas išreikštiniu, sistema nebėra besąlygiškai stabili.
