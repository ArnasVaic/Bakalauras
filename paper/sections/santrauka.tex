\sectionnonum{Santrauka}

Itrio aliuminio granato (YAG) sintezės reakcijos metu reagentai gali būti periodiškai maišomi, taip pagreitinant reakciją. Šiame darbe sudaryti kompiuteriniai YAG reakcijos-difuzijos sistemą sprendžiantys modeliai grįsti išreikštiniu bei ADI metodais, papildomai modeliuojamas maišymo procesas. Matematiškai parodyta modelio stabilumo sąlyga išreikštiniam modeliui, verifikuotas modelio korektiškumas, analizuotas modelių tikslumas ir efektyvumas. Darbe analizuojami du skirtingi maišymo modeliai -- atsitiktinis ir tobulas, analizuojami šių modelių rezultatai, modeliuojama įvairaus dydžio erdvė. Pastebėta, kad tobulo maišymo rezultatai pagreitina reakciją, o optimalaus išmaišymo laikas nepriklauso nuo modeliuojamos erdvės dydžio, didinant sritį optimali reakcijos trukmė konverguoja. Tuo tarpu atsitiktinio maišymo rezultatai neatitinka praktikoje pastebimo reakcijos pagreitėjimo, net ir modeliuojant įvairaus dydžio sritys.

\newpage 

\sectionnonum{Summary}

\newpage 