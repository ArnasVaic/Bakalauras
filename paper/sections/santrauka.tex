\section*{Santrauka}

Itrio aliuminio granato (YAG) sintezės reakcijos metu reagentai gali būti periodiškai maišomi, taip pagreitinant reakciją. Šiame darbe sudaryti kompiuteriniai YAG reakcijos-difuzijos sistemą sprendžiantys modeliai, pagrįsti išreikštiniu bei ADI metodais, papildomai įtraukiant maišymo procesą. Matematiškai nustatyta išreikštinio modelio stabilumo sąlyga, patikrintas modelio korektiškumas, įvertintas modelių tikslumas ir efektyvumas. Darbe nagrinėjami du skirtingi maišymo modeliai -- atsitiktinis ir tobulas, analizuojami jų rezultatai įvairaus dydžio erdvėse. Nustatyta, kad tobulo maišymo modelis atitinka realybėje pastebimą reakcijos pagreitėjimą, o optimalus maišymo momentas išlieka pastovus nepriklausomai nuo nagrinėjamos srities dydžio. Be to, didėjant sričiai, optimali reakcijos trukmė mažėja ir artėja į savo ribinę reikšmę. Tuo tarpu atsitiktinio maišymo rezultatai nesutampa su praktikoje pastebimu reakcijos pagreitėjimu, nepriklausomai nuo modeliuojamos srities dydžio.

\newpage 

\section*{Summary}

During the synthesis reaction of yttrium aluminum garnet (YAG), the reagents can be periodically mixed, thereby accelerating the reaction. In this work, computational models solving the YAG reaction-diffusion system were developed, based on explicit and ADI methods, additionally incorporating the mixing process. The stability condition of the explicit model was mathematically determined, the model's correctness was verified, and the accuracy and efficiency of the models were evaluated. Two different mixing models — random and perfect — are examined, and their results are analyzed in simulation domains of various sizes. It was found that the perfect mixing model results correspond to the reaction acceleration observed in practice, and the optimal mixing moment remains consistent regardless of the domain size. Furthermore, as the domain size increases, the optimal reaction time converges. In contrast, the results of the random mixing model do not match the reaction acceleration observed experimentally, regardless of the size of the simulated domain.

\newpage 

% raktažodžiai: YAG, Matematinis modeliavimas, reakcijos-difuzijos sistema, skaitiniai metodai, ADI, maišymo modeliavimas