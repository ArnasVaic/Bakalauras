\sectionnonum{Santrauka}

Itrio aliuminio granato (YAG) sintezės reakcijos metu reagentai gali būti periodiškai maišomi, taip pagreitinant reakciją. Šiame darbe sudaryti kompiuteriniai YAG reakcijos-difuzijos sistemą sprendžiantys modeliai, pagrįsti išreikštiniu bei ADI metodais, papildomai modeliuojant maišymo procesą. Matematiškai nustatyta modelio stabilumo sąlyga išreikštiniam modeliui, verifikuotas modelio korektiškumas, analizuotas modelių tikslumas ir efektyvumas. Darbe nagrinėjami du skirtingi maišymo modeliai -- atsitiktinis ir tobulas, vertinami jų rezultatai bei modeliuojamos įvairaus dydžio erdvės. Nustatyta, kad tobulo maišymo atveju reakcijos trukmė trumpėja -- modelis atitinka realybę, o optimalus maišymo momentas nepriklauso nuo modeliuojamos srities dydžio. Taip pat, didėjant sričiai, optimali reakcijos trukmė konverguoja. Tuo tarpu atsitiktinio maišymo rezultatai nesutampa su praktikoje stebimu reakcijos pagreitėjimu, nepriklausomai nuo modeliuojamos srities dydžio.

\newpage 

\sectionnonum{Summary}

\newpage 