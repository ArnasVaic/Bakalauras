\section{YAG reakcijos matematinis modeliavimas}

YAG sintezės metu vyksta elementari cheminė reakcija, kurios metu reaguoja aliuminio ir itrio oksidai:
\begin{align*}
  \ce{3Y_2O_3 + 5Al_2O_3 -> 2Y_3Al_5O_12}
\end{align*}

Prieš pradedant reakciją, metalų oksidai yra sutrinami iki smulkiagrūdžių miltelių ir maišomi, kol susidaro homogeniškas mišinys. Paruoštas oksidų mišinys nuolat kaitinamas aukštesnėje nei 1000\degree C temperatūroje. Esant tokiai temperatūrai, metalų oksidai lydosi ir vyksta medžiagų difuzija, todėl reakcijos produktas formuojasi mikrodalelių sienelėse. Priklausomai nuo pasirinktos temperatūros, gali kisti dalelių savybės — jų tūris, reakcijos ir difuzijos sparta. Eksperimentiniu būdu nustatyta, kad individualių mikrodalelių tūriai prie 1000\degree C temperatūros siekia apie $1\,\mu\text{m}^3$, o pakilus temperatūrai iki 1600\degree C — apie $10\,\mu\text{m}^3$ \cite{ivanauskasComputationalModellingYAG2009}. Minėtame šaltinyje autoriai taip pat pateikia reakcijos ir difuzijos konstantas prie skirtingų temperatūrų:

\begin{figure}[h]
  \centering
  \includegraphics[width=0.25\linewidth]{images/metal_oxides_mixture.png}
  \caption{Priartinto metalų oksidų mišinio iliustracija \cite{ivanauskasComputationalModellingYAG2009}}
  \label{fig:metal-oxides-mixuter}
\end{figure}

Verta paminėti, kad \ref{fig:metal-oxides-mixuter} pav. pateikta iliustracija nėra visiškai tiksli — žinoma, jog mikrodalelės išsidėsčiusios glaudžiai. Ivanauskas et al. \cite{ivanauskasModellingSolidState2005} pasiūlė modelį, kuriame mikrodalelės laikomos kubo arba kvadrato formos, priklausomai nuo dimensijos, kurioje modeliuojama reakcija. Taip pat laikoma, kad dalelės erdvėje yra periodiškai pasikartojančios, todėl pakanka modeliuoti mažą sritį, kurioje susiduria skirtingų medžiagų dalelės, kaip pavaizduota \ref{fig:periodic-space} pav.

\begin{figure}[h]
  \centering
  \includegraphics[width=0.6\linewidth]{images/periodic-space.png}
  \caption{Periodiškas reakcijos erdvės modelis}
  \label{fig:periodic-space}
\end{figure}

Vykstant reakcijai, chemikai periodiškai ištraukia reagentus iš krosnies, todėl maišymas atliekamas prie gerokai žemesnės temperatūros. Milteliai išmaišomi taip, kad nebūtų pažeista mikrodalelių struktūra — kitaip tariant, maišoma taip, jog mikrodalelės neskiltų.