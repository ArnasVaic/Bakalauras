\sectionnonum{Įvadas}

Itrio aliuminio granatas (YAG) yra kristalinis junginys, pasižymintis išskirtinėmis optinėmis savybėmis. Dėl šių savybių YAG kristalai, legiruoti neodimio jonais, plačiai naudojami kaip lazerių aktyviosios terpės. Tokie lazeriai taikomi įvairiose srityse, įskaitant odontologiją \cite{valentiUseErYAG2021}, pramoninę gamybą \cite{dubeyExperimentalStudyNd2008} ir daugelį kitų.

YAG gali būti sintezuojamas keliais skirtingais būdais, įskaitant kietafazę reakciją, solvoterminį procesą, nusodinimą, purškiamojo aerozolio terminį skilimą ir zolio-gelio metodą. Daugelis šių metodų yra chemiškai sudėtingi ir reikalauja specializuotos laboratorinės įrangos. Tarp jų kietafazė reakcija išsiskiria kaip paprasčiausias ir pramoninei gamybai tinkamiausias sintezės būdas \cite{zhangNovelSynthesisYAG2005}. Šios reakcijos metu itrio ir aliuminio oksidai reaguoja aukštoje temperatūroje, susidarant YAG kristalams. Praktikoje reakcijos greitį galima padidinti periodiškai maišant reagentus. Maišymo momentas yra itin svarbus eksperimento vykdymo parametras, nes nuo jo priklauso bendra reakcijos trukmė. Šis ryšys tarp reakcijos trukmės ir maišymo laiko nėra iki galo ištirtas, todėl šiame tyrime jis bus analizuojamas taikant kompiuterinį modeliavimą.

Kompiuterinis fizinių bei cheminių procesų modeliavimas leidžia giliau suprasti ir tiksliau prognozuoti šių procesų eigą bei jų rezultatus. Šis tyrimo metodas taikomas daugybėje sričių, pavyzdžiui matematiniam biosensorių modeliavimui \cite{baronasNonlinearEffectsDiffusion2017, baronasNonlinearEffectsPartitioning2024}. Šis metodas itin naudingas analizuojant YAG sintezės reakciją, kadangi laboratoriniai eksperimentai suteikia ribotas galimybes detaliai ištirti vykstančius mechanizmus. Nors kompiuterinis modeliavimas taikomas jau ilgą laiką, jis išlieka aktualus ir šiandien. Matematinį YAG reakcijos modelį pasiūlė Ivanauskas ir bendraautoriai \cite{ivanauskasModellingSolidState2005}, o susijusiuose darbuose \cite{ivanauskasComputationalModellingYAG2009,mackeviciusCloserLookComputer2012} eksperimentiniu būdu buvo nustatytos fizinės modelio konstantos, todėl kompiuterinių skaičiavimų rezultatus galima lyginti su eksperimentiniais duomenimis. Minėtuose straipsniuose daugiausia dėmesio skirta pačios YAG sintezės reakcijos eigai ir koeficientų nustatymui, todėl reakcija buvo modeliuojama labai mažoje fizinėje srityje, kurioje telpa vos kelios mikrodalelės. Šiame darbe tiriamas maišymo mechanizmo poveikis reakcijai, kuris gali priklausyti nuo fizinės srities dydžio, todėl bus modeliuojamos įvairaus dydžio sritys siekiant įvertinti, ar maišymo modelių rezultatai yra patikimi.

Šią reakciją aprašo trijų netiesinių parabolinių dalinių diferencialinių lygčių sistema, sudaranti reakcijos-difuzijos modelį. Žinoma, kad tokiose netiesinėse sistemose analitinis sprendinys dažniausiai neegzistuoja, todėl jų sprendimui taikomi skaitiniai metodai. Vienas dažniausiai naudojamų metodų yra baigtinių elementų metodas (FEM), leidžiantis suskaidyti nagrinėjamą sritį į mažesnius elementus ir diferencialines lygtis spręsti kiekviename jų. Taip pat gali būti taikomas ribinių elementų metodas (BEM), kuris naudoja tik ribinių sąlygų informaciją ir yra ypač efektyvus sprendžiant uždavinius su sudėtingomis geometrijomis.

Šiame darbe reakcijos-difuzijos sistemai spręsti taikysime du baigtinių skirtumų metodus (FDM). Pirmasis — išreikštinis metodas, dar žinomas kaip Oilerio integracija. Šis metodas pasižymi paprastu įgyvendinimu, tačiau yra sąlyginai stabilus ir nėra itin tikslus. Antrasis — ADI (neišreikštinis kintamos krypties) metodas, kuris yra techniškai sudėtingesnis, tačiau nesąlyginai stabilus ir užtikrina didesnį skaičiavimų tikslumą \cite{doi:10.1137/0103003}. Be to, šis metodas leis efektyviai modeliuoti dideles erdvines sritis. Nors ADI metodas sukurtas jau seniai, jis vis dar plačiai taikomas įvairiuose skaitiniuose tyrimuose \cite{gaidamauskaiteComparisonFiniteDifference2007}. Abu šie skaitiniai metodai išsamiai aprašyti literatūroje \cite{pressNumericalRecipes3rd2007,levequeFiniteDifferenceMethods2007}.

Šio \textbf{darbo tikslas} — sudaryti skaitinius YAG reakcijos modelius ir, išanalizavus jų rezultatus, nustatyti, kokį poveikį maišymas daro reakcijos trukmei. Siekiant šio tikslo, buvo suformuluoti šie uždaviniai:

\begin{enumerate}
  \item Sudaryti kompiuterinius YAG reakcijos modelius, taikant išreikštinį ir ADI skaitinius metodus.
  \item Verifikuoti skaitinių modelių rezultatus, palyginti jų tikslumą ir efektyvumą.
  \item Apibrėžti medžiagų maišymo modelius.
  \item Integruoti medžiagų maišymo modelius į skaitinius YAG reakcijos modelius.
  \item Analizuoti skaitinių modelių rezultatus siekiant nustatyti, ar maišymo modelių savybės išlieka pastovios skirtingo dydžio srityse.
  \item Įvertinti, kaip reakcijos trukmė priklauso nuo maišymo momento, remiantis skaitinių modelių rezultatais.
\end{enumerate}
