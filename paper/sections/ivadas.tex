\sectionnonum{Įvadas}

Itrio aliuminio granatas (YAG) yra kristalinis junginys, pasižymintis išskirtinėmis optinėmis savybėmis. Dėl šių savybių YAG kristalai, legiruoti neodimio jonais, plačiai naudojami kaip lazerių aktyvioji terpė. Tokie lazeriai taikomi įvairiose srityse, įskaitant odontologiją \cite{valentiUseErYAG2021}, pramoninę gamybą \cite{dubeyExperimentalStudyNd2008} ir daugelį kitų.

YAG gali būti sintezuojama keliais skirtingais būdais, įskaitant kietafazę reakciją, solvoterminį procesą, nusodinimą, purškiamo aerozolio terminį skilimą ir zolio-gelio metodą. Daugelis šių metodų yra chemiškai sudėtingi ir reikalauja specializuotos laboratorinės įrangos. Tarp jų kietafazė reakcija išsiskiria kaip paprasčiausias ir pramoninei gamybai tinkamiausias sintezės būdas \cite{zhangNovelSynthesisYAG2005}. Šios reakcijos metu itrio ir aliuminio oksidai reaguoja aukštoje temperatūroje, susidarant YAG kristalams. Praktikoje šios reakcijos greitį galima padidinti periodiškai maišant reagentus. Šis išmaišymo laikas labai svarbus eksperimento vykdymui, nes nuo jo priklauso reakcijos trukmė. Šis ryšys tarp reakcijos trukmės ir maišymo laiko nėra iki galo ištirtas, todėl šiame tyrime tai ir tirsime pasitelkdami kompiuterinį modeliavimą.

Kompiuterinis fizinių bei cheminių procesų modeliavimas yra plačiai taikomas tyrimo metodas, leidžiantis giliau suprasti ir tiksliau prognozuoti tokių procesų eigą bei jų rezultatus. Šis metodas itin naudingas YAG sintezės reakcijos analizėje, kadangi laboratoriniai eksperimentai suteikia ribotas galimybes detaliai ištirti vykstančius mechanizmus. Nors šis metodas taikomas jau ilgą laiką, jis išlieka aktualus ir šiandien. Matematinį YAG reakcijos modelį pasiūlė Ivanauskas et al. \cite{ivanauskasModellingSolidState2005}, o susijusiuose darbuose \cite{ivanauskasComputationalModellingYAG2009,mackeviciusCloserLookComputer2012} eksperimentiniu būdu nustatytos fizinės modelio konstantos, todėl kompiuterinių skaičiavimų rezultatus galima lyginti su eksperimentiniais duomenimis. Minėtuose straipsniuose dėmesys skiriamas pačiai YAG sintezės reakcijai ir koeficientų nustatymui ir dėl to reakcija yra modeliuojama labai mažoje fizinėje erdvėje, kurioje telpa vos kelios mikrodalelės. Šiame darbe yra tiriamas maišymo mechanizmo poveikis reakcijai, kuris gali kisti priklausomai nuo fizinės erdvės dydžio, todėl modeliuosime įvairaus dydžio erdves norint užtikrinti, kad maišymo modelių rezultatai yra patikimi.

Šią reakciją aprašo trijų netiesinių parabolinių dalinių diferencialinių lygčių sistema, sudaranti reakcijos-difuzijos modelį. Yra žinoma, kad tokiose netiesinėse sistemose analitinis sprendinys įprastai neegzistuoja, todėl jų sprendimui taikomi skaitiniai metodai. Vienas iš dažniausiai naudojamų metodų yra FEM - jis leidžia suskaidyti nagrinėjamą sritį į mažesnius elementus ir spręsti diferencialines lygtis kiekviename iš jų. Taip pat gali būti taikomas BEM, kuris naudoja tik ribinių sąlygų informaciją, todėl yra efektyvus sprendžiant uždavinius su sudėtingomis geometrijomis.

Šiame darbe reakcijos-difuzijos sistemai spręsti taikysime du FDM metodus. Pirmasis - išreikštinis FTCS metodas, dar žinomas kaip Oilerio integracija. Šis metodas pasižymi paprasta implementacija, tačiau yra sąlyginai stabilus ir nėra itin tikslus. Antrasis metodas - ADI, kuris yra techniškai sudėtingesnis, tačiau nesąlyginai stabilus ir užtikrina didesnį tikslumą \cite{doi:10.1137/0103003}. Taip pat šis metodas mum leis efektyviai modeliuoti dideles reakcijos erdves. Nepaisant to, kad ADI metodas sukurtas jau seniai, jis vis dar plačiai naudojamas \cite{gaidamauskaiteComparisonFiniteDifference2007}. Šie skaitiniai metodai detaliai aprašyti literatūroje \cite{pressNumericalRecipes3rd2007,levequeFiniteDifferenceMethods2007}.

Šio \textbf{darbo tikslas} yra sudaryti skaitinius YAG reakcijos modelius ir ištirti jų rezultatus norint išsiaiškinti kokį poveikį maišymas turi reakcijos trukmei. Šiam tikslui pasiekti buvo iškelti šie uždaviniai:

% Šio darbo tikslas yra sudaryti kompiuterinę difuzijos-reakcijos sistemą, kuri modeliuotų YAG reakciją bei ištirti maišymo poveikį YAG kristalų sintezės spartai modeliuojant reakcija įvairaus dydžio erdvėse. Iškelti darbo uždaviniai:

\begin{enumerate}
  \item Sudaryti kompiuterinius YAG reakcijos modelius, taikant išreikštinį FTCS ir ADI skaitinius metodus.
  \item Verifikuoti skaitinių modelių rezultatus, palyginti jų tikslumą ir efektyvumą.
  \item Apibrėžti medžiagų maišymo modelius.
  \item Integruoti medžiagų maišymo modelius į skaitinius YAG reakcijos modelius.
  \item Analizuoti skaitinių modelių rezultatus, siekiant nustatyti, ar maišymo modelių savybės skirtingo dydžio erdvėse išlieka pastovios.
  \item Įvertinti, kaip reakcijos trukmė priklauso nuo maišymo momento, remiantis skaitinių modelių rezultatais.
\end{enumerate}